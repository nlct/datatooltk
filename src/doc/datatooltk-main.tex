
\usepackage{fontspec}
\setmainfont{Linux Libertine O}
\usepackage{verbatim}
\usepackage
 [
    novref,
%   debug=showwrgloss
 ]{texjavahelp}

\hypersetup{colorlinks,linkcolor=blue}
\DeclareGraphicsExtensions{.png,.pdf,.tex}

\glsxtrnewgls{plugin.}{\plugin}
\glsxtrnewgls{template.}{\template}

\glsxtrnewgls{opt.action.}{\actionopt}
\newcommand{\actionoptval}[2]{\actionopt{#1}\dequals #2}
\newcommand{\actionoptvalm}[2]{\actionopt{#1}\dequals \marg{#2}}

\newcommand{\appname}{DatatoolTk}

\title{User Manual for \appname}
\author{Nicola L.C. Talbot\\\href{https://www.dickimaw-books.com/}{\nolinkurl{dickimaw-books.com}}}

\InputIfFileExists{version}{}{}

\GlsXtrLoadResources[src={datatooltk,\langbibsrcs},
 \TeXJavaHelpSymbolResourceOptions
]

\GlsXtrLoadResources[src={datatooltk,\langbibsrcs},
 \TeXJavaHelpGlsResourceOptions
]

\begin{document}
\maketitle
\frontmatter
\tableofcontents
\listoffigures

\mainmatter
\chapter{Introduction}
\label{sec:introduction}

The \LaTeX\ \sty{datatool} package is able to save databases in its
own internal format to allow for rapid loading (using
\code{\gls+{DTLwrite}\oarg{\optval{format}{dbtex}}\margm{file}} with
\sty{datatool} v3.0 or, for earlier versions,
\code{\gls+{DTLsaverawdb}\margm{file}} or
\code{\gls+{DTLprotectedsaverawdb}\margm{file}}).  Files in this
\ext{dbtex} format are difficult to edit and only a \TeX pert should
attempt it, but they are by far the fastest way of loading a
\sty{datatool} \gls{database} in \LaTeX. The \app{datatooltk}
application provides a \gls{gui} making it easier to edit these
files. 

\appname\ can also import data from \ext{csv} files, from
Excel \ext{xls} (but not \ext{xlsx})
or Open Document \ext{ods} spreadsheets, from 
\gls{sql} databases and from \sty{probsoln} databases.
This manual assumes the user has some knowledge of the
\sty{datatool} package. Please ensure you have at least version
2.15 of \sty{datatool} installed in your \TeX\ distribution.
(Although the latest version is recommended.)

The \app{datatooltk} application can be run in either 
batch mode (default) or \gls{gui} (see \sectionref{sec:gui}).

\section{Command Line Options}
\label{sec:commandoptions}

The \appname\ application may be run from the command line in batch
mode:
\appdef{datatooltk}
In this case, a load or import option must be used (such as
\switch{in} or \switch{csv}), as well as an output file specified
with \switch{output}. The application may also be run with a
graphical interface:
\appdef{datatooltk-gui}
This is essentially equivalent to
\begin{terminal}
\app{datatooltk} \switch{gui} \oargm{options}
\end{terminal}
but additionally has a splash screen. In \gls{gui} mode, the output
file should not be specified, and the load\slash import switches may
be omitted. The \menu{file} menu may be used to load, import and
save files.

Available options for the command line invocation are listed below.
Note that some of the default values may be changed through the
\gls{gui}, which saves your preferences.  These will be picked up by
the next batch invocation.

You can't combine any of the load/import options: \switch{in},
\switch{csv}, \switch{xls}, \switch{ods}, \switch{sql},
\switch{probsoln}. You also can't combine any of the merge options:
\switch{merge}, \switch{merge-csv}, \switch{merge-xls},
\switch{merge-ods}, \switch{merge-sql}, \switch{merge-probsoln}.
The merge import options use the same settings as the import
options. If you want to merge, for example, sheet~1 and sheet~2 from
the same spreadsheet, you will have to first import one or both of
them to a \ext{dbtex} file and then perform the merge.  With the
exception of \switch{merge-sql}, all the merge options will ignore a
missing file and just print a warning to STDERR.

The import functions are one-way. You can't export back to any of 
those formats.

\switchdef{gui}
Invoke \app{datatooltk} in \gls{gui} mode.

\switchdef{batch}
Invoke \app{datatooltk} in batch mode (default).

\switchdef{output}
Save the database to \meta{filename} (batch mode only).
To guard against accidentally overwriting a document file, \app{datatooltk}
now forbids the \ext{tex} extension for output files. See
\sectionref{sec:fileext}.

\switchdef{output-format}
Sets the default file format for output files. The value may be one
of (case-insensitive): \qtt{DBTEX 2.0}, \qtt{DBTEX 3.0}, 
\qtt{DTLTEX 2.0}, \qtt{DTLTEX 3.0}. The value may also be an empty
string or \qtt{default} to indicate the default format should be
used. The default will either match the format of the input file or,
if data was imported, \qtt{DBTEX 3.0}. Alternatively, this setting
can be saved in the \widget{preferences.tex} tab of the \dialog{preferences}
dialog box (see \sectionref{sec:preferences.tex}).

\begin{information}
If the \gls{DBTEX3} format is used, there are further settings to
determine whether or not to store the entry value in
\sty{datatool}['s] datum format. See
\sectionref{sec:preferences.tex} for these settings.
\end{information}

\switchdef{in}
Load \meta{datatool file}. The switch \switch{in}
(or \swch{in}) is optional, so:
\begin{terminal}
\app{datatooltk} \meta{file}
\end{terminal}
is equivalent to 
\begin{terminal}
\app{datatooltk} \switch{in} \meta{file}
\end{terminal}
See \sectionref{sec:fileext} for permitted file types.

\switchdef{tex-encoding}
Set the encoding for the \TeX\ (\ext{tex}, \ext{dtltex} and
\ext{dbtex}) files to \meta{encoding}. The \meta{encoding} may be
the keyword \qtt{default} or an empty string to indicate Java's
default file encoding (see \sectionref{sec:fileext}). Make sure that
your \LaTeX\ document matches the given \meta{encoding}. The
encoding of \gls{csv} files is independent of the \TeX\
(\ext{dbtex}) encoding and is set through \switch{csv-encoding}.

\switchdef{name}
If used with \switch{in}, \switch{csv}, \switch{sql}, \switch{xls}, \switch{ods}
or \switch{probsoln}, the \switch{name} switch sets the database label to 
\meta{name}. (See \sectionref{sec:fileext}.)

\switchdef{version}
Print the version details to STDOUT and exit.

\switchdef{help}
Print a brief summary of available options to STDOUT and exit.

\switchdef{debug}
Enable debug mode.

\switchdef{nodebug}
Disable debug mode. (Default.)

\switchdef{compat} \meta{level}
Set the compatibility level. The argument may be the keyword
\qtt{latest} (no backward compatibility required) or \qtt{1.6}
(compatible with version 1.6 and below).  The \code{\switch{compat}
1.6} setting only affects \switch{shuffle} and is provided for old
documents.

\switchdef{owner-only}
Set read/write permissions when saving \ext{dbtex} files to owner only.
(Has no effect on some operating systems.)

\switchdef{noowner-only}
Don't change read/write permissions when saving \ext{dbtex} files.

\switchdef{map-tex-specials}
Map \TeX\ special characters when importing data from 
\gls{csv} or \gls{sql}.

\switchdef{nomap-tex-specials}
Don't map \TeX\ special characters when importing data from
\gls{csv} or \gls{sql}. (Default.)

\switchdef{auto-trim-labels}
Automatically strip leading and trailing spaces from database 
and column identifiers. (See also \sectionref{sec:spaces}.)

\switchdef{noauto-trim-labels}
Don't automatically strip leading and trailing spaces from database 
and column identifiers.

\switchdef{seed}
Set the random generator seed to \meta{number} or clear it
if \meta{number} is the empty string \code{""}. (See \sectionref{sec:shuffle}.)

\switchdef{shuffle}
Shuffle the database. (Shuffle is always performed after sort,
regardless of the option order.)

\switchdef{noshuffle}
Don't shuffle the database. (Default.)

\switchdef{sort}
Sort the database according to the column whose label is
\meta{field}. Optionally, \meta{prefix} may be the plus character
\qtt{+} (ascending order) or the hyphen\slash minus character
\qtt{-} (descending order).  If \meta{prefix} is omitted, ascending
is assumed.  (See \sectionref{sec:sort}.)

\switchdef{sort-locale}
If the \meta{value} is the keyword \qtt{none} use letter-sorting
for strings. That is, compare the Unicode values of each character.
Otherwise \meta{value} should be a valid \gls{ietf} 
language tag that identifies a locale. Strings will then be sorted 
according to that locale's alphabetical order. Note that \app{datatooltk} can't 
interpret \LaTeX\ commands. (If you need that ability, you might want 
to consider using \app{bib2gls} with \sty{glossaries-extra} 
instead.) The default setting is to use a letter-sort for strings.

This setting also governs the string comparison functions used by 
the filter option \switch{filter}.

\switchdef{sort-case-sensitive}
(Ignored with locale-sensitive comparisons.)
Use case-sensitive comparison when letter-sorting strings.
This setting also governs the string comparison functions
used by the filter option \switch{filter}.

\switchdef{sort-case-insensitive}
(Default.) Use case-insensitive comparison when letter-sorting strings.
This compares the lower case versions of the strings using a
character code comparison. This setting also governs the string
comparison functions used by the filter option \switch{filter}.

\switchdef{truncate}
Truncate the database to the first \meta{n} rows.
(Has no effect if \meta{n} is greater than or equal
to the total number of rows.) Truncation is always performed
after any sorting, shuffling and filtering, but before column 
removal.

\switchdef{remove-columns}
Remove the columns identified by \meta{column list}, which
may be a comma-separated list of column labels (for example, 
\qtt{Details,Comments}) or a comma-separated list of column indexes
or ranges (where the first column has the index 1). For example,
\qtt{3,5-7,9} indicates columns 3, 5, 6, 7 and 9. You can't mix
labels and indexes, so \qtt{Details,5,Comments} would mean the
columns identified by the labels \qtt{Details} \qtt{5} (which
might not correspond to the fifth column) and \qtt{Comments}.
Ranges are only permitted with indexes and may be open ended. For 
example, \code{-4} is equivalent to \code{1-4} and \code{3-} means
the third column onwards. This option is not cumulative and
may not be used with \switch{remove-except-columns}. Column removal
is always performed last, so you can still use \app{datatooltk} to sort or merge 
by a column that isn't required in the document.

\switchdef{remove-except-columns}
This option is similar to \switch{remove-columns} but the
\meta{column list} indicates which columns to keep. All other
columns are removed. The argument has the same syntax as for
\switch{remove-columns}. This option is not cumulative and 
may not be used with \switch{remove-columns}. Column removal
is always performed last, so you can still use \app{datatooltk} to sort or merge 
by a column that isn't required in the document.

\switchdef{filter}
Adds the given filter. This filter returns true if the value in
the column whose label is given by \meta{key}
matches the operation \meta{operator} \meta{value}
where \meta{operator} may be one of:
\begin{itemize}
\item \code{eq} (equals \meta{value}),
\item \code{ne} (does not equal \meta{value}),
\item \code{le} (less than or equal to \meta{value}),
\item \code{lt} (less than \meta{value}),
\item \code{ge} (greater than or equal to \meta{value}),
\item \code{gt} (greater than \meta{value}),
\item \code{regex} (matches the regular expression \meta{value}).
\end{itemize}
Multiple filters may be used. The regular expression
should be in the format used by
\href{http://docs.oracle.com/javase/7/docs/api/java/util/regex/Pattern.html}{java.util.regex.Pattern}.
Filtering is always performed after sorting and shuffling.
Numerical comparisons are used for columns that are identified 
as either integer or real data types otherwise string
comparisons are used, except in the case of 
\code{regex} where the data type is disregarded
and all values are assumed to be strings. (If the column type
is identified as having an integer or real data type but
\meta{value} is not a number, a string comparison will be used.)
For example: \code{\switch{filter} Level eq 3}
means that the filter should return true if the value in the column
whose label is \qtt{Level} is equal to 3.
If there isn't a column with the label \meta{key}, 
a warning is printed on the standard error stream and the
filter is ignored.

\switchdef{filter-or}
Use OR operator when filtering. (Default.)
This has no effect if you only supply one filter.

\switchdef{filter-and}
Use AND operator when filtering. 
This has no effect if you only supply one filter.

\switchdef{filter-include}
When filtering, discard rows that don't match the filter (and keep
those that do match). This is the default action.

\switchdef{filter-exclude}
When filtering, discard rows that match the filter (and keep those that don't match).

\switchdef{merge}
Merges the input or imported database with the database stored
in the given \meta{db file}. Each row
in \meta{db file} is merged with the row that has a matching
value in the column whose label is given by \meta{key}.
Both databases must contain a column with that label. (Each entry
in that column should ideally be unique.)
If no matching row is found, a new row is added. If both databases
share additional columns, the values in \meta{db file}
override those in the original database. If \meta{db file}
doesn't exist, a warning is issued and the option is ignored.
This option is always implemented before any sorting, shuffling,
filtering or truncating. Only one of the merge options is permitted.

\switchdef{csv}
Import data from the given \gls{csv} file.
(See \sectionref{sec:importcsv}.)

\switchdef{merge-csv}
As \switch{merge} but the data to be merged is imported from the given \gls{csv} file.

\switchdef{csv-sep}
Specify the character used to separate values in the 
\gls{csv} file. (Defaults to a \gls{comma} character.)

\switchdef{csv-delim}
Specify the character used to delimit values in the 
\gls{csv} file. (Defaults to a \gls{dbqt} character.)

\switchdef{csv-skiplines}
Skip the first \meta{n} rows of the \switch{csv} file. 
(Useful if you have a comment block at the start of the file that
needs to be skipped.)
The value \meta{n} may be 0 (don't skip) or a positive integer 
indicating the number of rows to skip. Blank rows are always included
in this count, even if \switch{csv-skip-empty-rows} is set.
The spreadsheet import functions also use this setting.

\switchdef{csv-strictquotes}
Ignore any undelimited information (where the delimiter is given 
by \switch{csv-delim}).

\switchdef{nocsv-strictquotes}
Allow undelimited data.

\switchdef{csv-encoding}
Set the encoding for the \gls{csv} files to \meta{encoding}. The
\meta{encoding} may be the keyword \qtt{default} or an empty
string to indicate the default.
The encoding of the \TeX\ (\ext{dbtex}) file is independent of the 
\gls{csv} encoding and is set through \switch{tex-encoding}.

\switchdef{csv-header}
The \gls{csv} file has a header row. (Default.) The spreadsheet import functions
also use this setting.

\switchdef{nocsv-header}
The \gls{csv} file doesn't have a header row. The spreadsheet import functions
also use this setting.

\switchdef{csv-skip-empty-rows}
Skip empty lines found in the \gls{csv} file.
(Spreadsheet import also uses this setting.)

\switchdef{nocsv-skip-empty-rows}
Don't skip empty lines found in the \gls{csv} file.
(Spreadsheet import also uses this setting.)

\switchdef{csv-escape}
Set the \gls{csv} file escape character to 
\meta{character}.  If your data includes the delimiter
character, you need to escape that character to prevent it from being
mistaken for the delimiter. For example, if the delimiter is 
the \gls{dbqt} character and the escape character is the \gls{bksl}
character, then a row of data may appear as:
\begin{verbatim}
12345,"A \"sample\" entry."
\end{verbatim} 
(This won't actually render properly in \LaTeX\ as won't produce the
typographically correct double quotes.) By default the \gls{csv}
escape character is the \gls{bksl} character (as in the above
example) which means that you must double the backslash if you have
any (La)TeX commands within the file.  To avoid this, you can set
the escape character to something else that doesn't occur in your
data.

Note that delimiter character is often also the escape character, so
the row of data would be:
\begin{verbatim}
12345,"A ""sample"" entry."
\end{verbatim} 

\switchdef{nocsv-escape}
Don't have an escape character for your \gls{csv} file.

\switchdef{xls}
Import data from the given Excel \ext{xls} file. 
(See \sectionref{sec:importspread}.)

\switchdef{merge-xls}
As \switch{merge} but the data to be merged is imported from the
given Excel \ext{xls} file.

\switchdef{ods}
Import data from the given Open Document Spreadsheet \ext{ods} file. 
(See \sectionref{sec:importspread}.)

\switchdef{merge-ods}
As \switch{merge} but the data to be merged is imported from the
given Open Document Spreadsheet \ext{ods} file.

\switchdef{sheet}
The sheet to select from the Excel workbook or Open Document 
Spreadsheet. This may either be an index (starting from 0) or 
the name of the sheet. If this option is omitted, the first sheet is assumed.

\switchdef{sql}
Import data from an \gls{sql} database where 
\meta{statement} is an \gls{sql} SELECT statement.
(See \sectionref{sec:importsql})

\switchdef{merge-sql}
As \switch{merge} but the data to be merged is imported using the
given \gls{sql} SELECT statement.

\switchdef{sqldb}
The \gls{sql} database name.

\switchdef{sqlprefix}
The Java \gls{sql} prefix. (Default: \qtt{jdbc:mysql://}.)
Currently, only \gls{mysql} is supported. Additional libraries will
be required for other \gls{sql} databases.

\switchdef{sqlport}
The \gls{sql} port number. (Default: 3306.)

\switchdef{sqlhost}
The \gls{sql} host. (Default: \qtt{localhost}.)

\switchdef{sqluser}
The \gls{sql} user name.

\switchdef{sqlpassword}
The \gls{sql} password (insecure). If omitted, you will be prompted 
for a password if you try to import data from an \gls{sql} 
database.

\switchdef{wipepassword}
For extra security, wipe the password from memory as soon as it 
has been used to connect to an \gls{sql} database. (Default.)

\switchdef{nowipepassword}
Don't wipe the password from memory as soon as it 
has been used to connect to an \gls{sql} database.

\switchdef{noconsole-action}
If in batch mode and a SQL password is required
and \switch{sqlpassword} hasn't been used, the
default action is for \app{datatooltk} to request a password
via the console. If there is no console available
the action is determined by \meta{action}
which may be one of:
\begin{itemize}
  \item \qtt{error} issue an error;
  \item \qtt{stdin} request the password via the standard
  input stream (less secure than using a console, and can produce 
  an annoying flicker);
  \item \qtt{gui} display a dialog box in which to enter the password (default).
\end{itemize}

\switchdef{probsoln}
Import \sty{probsoln} data from \meta{filename}.
(See \sectionref{sec:importprobsoln}.)

\switchdef{merge-probsoln}
As \switch{merge} but the data to be merged is imported from the
given \sty{probsoln} data set file.

\section{What it isn't}
\label{sec:notaspreadsheet}

The \app{datatooltk} application isn't intended to have the full functionality of
a spreadsheet. Its purpose is to allow you to edit 
\sty{datatool} databases with multilined entries. If your data 
just consists of numbers or short single-lined text, then you'll probably 
be better off just using a spreadsheet to input the data and 
use \app{datatooltk} in batch mode to convert from \gls{csv} to 
a \sty{datatool} file.

\section{DBTEX and DTLTEX Files}
\label{sec:fileext}

In batch mode, an input file can be specified with the \switch{in}
switch, or by simply providing the file name without a preceding switch.
In \gls{gui} mode, an input file can be opened with the
\menu{file.open} menu item. Only \ext+{dbtex} and \ext+{dtltex} files can be
input. For other file types, you will need to use the relevant
import option.

In batch mode, an output file must be specified with the
\switch{output} (or \swch{output}) switch. In \gls{gui} mode, a file
can be saved with the \menu{file.save} or \menu{file.save_as} menu
items. The output format can be specified with the
\switch{output-format} switch or by selecting the appropriate file
filter in the \menu{file.save_as} file selector. If you use
\menu{file.save}, the format will match the input file.

The output file encoding will default to Java's default
encoding, which usually matches the operating system's default, but
can be changed with the Java \code{file.encoding} property (see the
Java documentation for further details). Alternatively, you can use
the \switch{tex-encoding} switch or the selector in the
\widget{preferences.tex} tab of the \dialog{preferences}
dialog box (see \sectionref{sec:preferences.tex}).

Supported file formats for both input and output files are:
\gls{DBTEX2}, \gls{DBTEX3}, \gls{DTLTEX2} and \gls{DTLTEX3}.
These formats can be both input and saved in a document
using \sty{datatool}['s] \gls{DTLread} and \gls{DTLwrite} commands.

\begin{information}
As from \sty{datatool} v3.0, \code{\gls{DTLsaverawdb}\margm{file}} has been deprecated
in \manmsg{favour} of
\code{\gls{DTLwrite}\oarg{\optvalm{format}{dbtex-2}}\margm{file}}.
\end{information}

Although the DBTEX and DTLTEX files are all \LaTeX\ files, the file 
extensions are assumed to be \ext+{dbtex} and \ext+{dtltex} to
reduce the chances of accidentally overwriting the main document
file.

\appname\ will parse the first line of the input file to see if it
matches the format:
\begin{compactcodebox}
\% \meta{format} \meta{encoding}
\end{compactcodebox}
where \meta{format} is either \code{DBTEX} or \code{DTLTEX} and
\meta{encoding} is either the canonical encoding name (such as
UTF-8) or the \sty{inputenc} label (such as \optfmt{utf8}).
If the first line doesn't match this format, \gls{DBTEX2} is
assumed with \app{datatooltk}['s] default encoding (see
\switch{tex-encoding} and \sectionref{sec:preferences.tex}).

The \app{datatooltk} input function will first try parsing the file
according to the detected format.  If this fails, \app{datatooltk}
will retry using the \TeX\ Parser Library, which has some limited
understanding of \TeX\ syntax and \sty{datatool} commands.
This is more sophisticated than a simple pattern match, but it's slower.
For example, suppose the file contains:
\verbatiminput{../tests/test-dtlnewdb.tex}
This doesn't match the \ext{dbtex} format, so \app{datatooltk} will
switch to using the \TeX\ Parser Library, which can interpret these
commands, but note that if the file is subsequently saved
\strong{it will be saved in the \ext{dbtex} format}. Since this is
potentially dangerous (as there's a possibility that you might
accidentally load your document \ext{tex} file) if the 
\TeX\ Parser Library is used to load the file then, in \gls{gui}
mode, the filename for that database will be unset, which will 
trigger the \dgls{file.save_as} dialog if you try to save the database.

The \sty{datatool} database files loaded and saved by \app{datatooltk}
are just \LaTeX\ files, so they could simply have the standard
\ext{tex} extension, but to help differentiate the database files
from other files containing \TeX\slash \LaTeX\ code (such as picture-drawing
code), \app{datatooltk} assumes a default extension of \ext{dbtex}.
If you use this extension, remember to include it in the argument of
\gls{input} or \gls{DTLloaddbtex}.

\begin{information}
To guard against accidentally overwriting a document file, \app{datatooltk}
now forbids the \ext{tex} extension for output files.
\end{information}

Note that the database label (as used in commands like \gls{DTLnewdb})
is independent of the file name (although when importing data, it defaults 
to the file base name).  The database label can be changed using 
\menu{edit.edit_dbname} in 
\gls{gui} mode or via the command line option 
\code{\switch{name} \meta{label}}.

\FloatFig[htb]
{fig:setnamedialog}
{\includeimg
 [scale={0.75},alt={Image of dialog box}]
 {images/setnamedialog.png}
}
{Setting the Database Name}

\begin{example}
 {Loading and Displaying Data in the Document}
 {ex:loadanddisplay}

Suppose you have a database file called \filefmt{my-data.dbtex}
and you have set the database label to just \qtt{data}
(as shown in \figureref{fig:setnamedialog}).
Then you can load and display the data using:
\begin{codebox}
\cmd{documentclass}\marg{article}
\cmd{usepackage}\marg{datatool}\comment{remember to load the datatool package}
\codepar
\gls{input}\marg{my-data.dbtex}\comment{load the database from file `my-data.dbtex'}
\codepar
\cbeg{document}
\gls{DTLdisplaydb}\marg{data}\comment{Display the database identified by the name `data'}
\cend{document}
\end{codebox}

If you can't remember the name you assigned to the database, you can 
access it using \gls{dtllastloadeddb}.
\begin{codebox}
\cmd{documentclass}\marg{article}
\cmd{usepackage}\marg{datatool}\comment{remember to load the datatool package}
\codepar
\gls{input}\marg{my-data.dbtex}\comment{load the database from file `my-data.dbtex'}
\codepar
\cbeg{document}
\gls{DTLdisplaydb}\marg{\gls{dtllastloadeddb}}\comment{Display the last loaded database}
\cend{document}
\end{codebox}

Alternatively, as from \sty{datatool} version 2.20, use 
\gls{DTLloaddbtex} instead of \gls{input}:
\begin{codebox}
\cmd{documentclass}\marg{article}
\cmd{usepackage}\marg{datatool}\comment{remember to load the datatool package}
\codepar
\gls{DTLloaddbtex}\marg{\cmd{mydata}}\marg{my-data.dbtex}\comment{load the database from file `my-data.dbtex'}
\codepar
\cbeg{document}
\gls{DTLdisplaydb}\marg{\cmd{mydata}}\comment{Display the database}
\cend{document}
\end{codebox}
If you have at least version 3.0 of \sty{datatool}, you can use
\gls{DTLread} instead of \gls{input} or \gls{DTLloaddbtex} but remember to set the
\opt{format} option to \code{dbtex}:
\begin{codebox}
\gls{DTLread}\oarg{\optval{format}{dbtex}}\marg{my-data}\comment{load the database from file `my-data.dbtex'}
\end{codebox}
With DBTEX v2.0, you can't change the database name in the document,
but you can with DBTEX v3.0:
\begin{codebox}
\cmd{documentclass}\marg{article}
\cmd{usepackage}\marg{datatool}\comment{v3.0+}
\codepar
\gls{DTLsetup}\marg{\optval{default-name}{mydata}}
\gls{DTLread}\oarg{\optval{name}{mydata},\optval{format}{dbtex}}\marg{my-data}
\codepar
\cbeg{document}
\gls{DTLaction}\marg{display}\comment{Display the database}
\cend{document}
\end{codebox}
\end{example}

\section{Verbatim}
\label{sec:verbatim}

Since the contents of the database are stored in a \TeX\ token
register, and assigned to control sequences via commands like
\gls{DTLforeach}, verbatim text is not permitted. This is 
a common problem when attempting to use verbatim text within
a command and is covered in the UK List of \TeX\ Frequently
Asked Questions (\href{https://texfaq.org/FAQ-verbwithin}{Why doesn't verbatim work within...?}).
The \app{datatooltk} application checks for verbatim text when you load a database or import
data (unless the \qt{map \TeX\ special characters} property is set for
\gls{csv} or \gls{sql} imports). Also, \app{datatooltk} checks for
verbatim text when you edit the contents of a cell. If it
detects any, it will give a warning. If you ignore the warning, 
\TeX\ will give an error if you then attempt to load the database into
a document.

\begin{information}
The verbatim check consists of checking for any occurrences of
\gls{verb}, \gls{lstinline} or the beginning of the \env{verbatim},
\env{lstlisting} or \env{alltt} environments. Other verbatim-like
commands or environments are not recognised.
\end{information}

If you just have a short fragment of inline verbatim text, consider
one of the alternatives listed in \href{https://texfaq.org/FAQ-verbwithin}{the FAQ}.
If on the other hand you have a block of verbatim text you'll have 
to put the verbatim text in a separate file
and load it using \gls{verbatiminput} (from the 
\sty{verbatim} package) or \gls{lstinputlisting} (from the
\sty{listings} package). For example, in \figureref{fig:exverb} 
I have used \gls{lstinputlisting}.

\FloatFig[htb]
{fig:exverb}
{%
  \includeimg
  [
   scale=0.75,
   alt={Image of main window showing cells with listing input}
  ]
  {images/verb.png}%
}
{Verbatim Blocks Need to be in Separate Files}

That database requires two files: \filefmt{HelloWorld.java}
\verbatiminput{../tests/HelloWorld.java}
and \filefmt{HelloUser.java}:
\verbatiminput{../tests/HelloUser.java}

Assuming that I've saved my database in a file called
\filefmt{prob-verb.dbtex} with database label \qtt{problems},
here's a sample document:
\verbatiminput{../tests/test-probsoln-doc.tex}

\seealsorefs{sec:shuffle,sec:sortshuffle,sec:importprobsoln}

\section{Leading/Trailing Spaces}\label{sec:spaces}

Spaces characters at the start and end of cell contents and column titles 
are removed when writing the \ext{dbtex} file (but they will still show
in the \gls{gui} until the file is saved and reloaded).

Leading space characters are naturally ignored in \sty{datatool}'s 
internal format. For consistency, \app{datatooltk} now trims leading and trailing 
spaces from each cell and column title when writing the \ext{dbtex} 
file as they are usually unwanted and easy to miss. If you explicitly want
a space you need to use \LaTeX\ markup, such as \gls{space}. 
A trailing space can also be hidden by a final comment. For
example, with the cell contents set to:
\begin{codebox}
\cmd{space} text \comment{comment}
\end{codebox}
then there will be a leading and trailing space if the value is used in the 
document. Any space characters at the end of the comment line will be 
stripped, but those would naturally be ignored by \LaTeX\ anyway.
Remember that blank lines are converted to \gls{DTLpar}, 
so paragraph breaks won't be recognised as spaces by the trimming code.
Column and database \meta{labels} will only have leading and
trailing spaces stripped if the auto trim labels setting is on.
Column titles are always trimmed when writing the \ext{dbtex} file.

\section{Null Values}\label{sec:null}

Empty entries aren't the same as null entries. If you want 
a null entry, set the entry to \gls{@dtlnovalue}. 
A convenient way to do this is to select the cell and use
\menu{edit.cell_to_null}. Alternatively, you can set all entries in 
a selected column to null with \menu{edit.column.column_to_null}
and similarly for a selected row with \menu{edit.row.row_to_null}.

In your \LaTeX\ document, you can check for null values using 
\sty{datatool}'s \gls{DTLifnull} command. To check for empty 
values you can use one of \sty{etoolbox}'s commands, such as 
\gls{ifdefempty}. As from \sty{datatool} version 2.20, you
can also use \gls{DTLifnullorempty}.

\chapter{Graphical Mode}\label{sec:gui}

To run \app{datatooltk} in graphical mode you must invoke it with either 
\app{datatooltk-gui} or \code{\app{datatooltk} \switch{gui}}. The main window
is shown in \figureref{fig:mainwindow}. Each database is
in a tabbed pane, with the name of the database in the
tab. Note that the name corresponds to the database's identifying
label, as used in commands like \gls{DTLnewdb}. This is 
not necessarily the same as the filename (see \sectionref{sec:fileext}). 
Since this name is used as a label, it shouldn't contain any of 
\TeX's special characters or any other active characters that could 
cause problems.  An asterisk \code{*} following the label in the tab 
indicates that the database has been modified. If you move the mouse
over the tab, you will see the full pathname appear in a tooltip,
if the database has been saved to a \sty{datatool} file,
and the filename (without the path) will be shown in the title bar.

\FloatFig[htb]
 {fig:mainwindow}
 {%
    \includeimg
     [ scale=0.75, alt={Image of main window} ]
     {images/mainwindow.png}
 }
 {Main Window}

 You can use the \menu{file} menu to create a new 
database, load an existing database or import data (see 
\sectionref{sec:import}). To load an existing database, use
\menu{file.open}. These database files contain \LaTeX\ code
in a specific format. The \app{datatooltk} application assumes 
a \ext{dbtex} file extension (see \sectionref{sec:fileext}).
You can load these files into a \LaTeX\ document using 
\gls{input} or \gls{DTLloaddbtex}, but remember to specify 
the \ext{dbtex} extension. (Also remember to load the 
\sty{datatool} package.)

Each column has a corresponding data type: string, integer, real
or currency. The type is automatically detected from the column data,
but can be changed, as described in \sectionref{sec:editheader}.

Non-string entries can be edited by double-clicking on the relevant cell, 
or you can select the relevant cell and use \menu{edit.edit_cell}.
In the first case, a cursor will appear in the cell and you 
can edit the numerical value and press \qt{Enter} to finish 
editing. In the second case, the cell editor dialog box will 
open, see \sectionref{sec:celleditor}.

\begin{information}
Only the first few lines of a string entry are visible in the main 
window. If an entry has more than that number of lines, you will need to 
use the cell editor dialog box to view the entire contents of that 
cell.
\end{information}

The default row height can be changed in the
\dialog{preferences} dialog box (see \sectionref{sec:preferences}).
Columns set to integer or real data types have single-lined cells with no
line wrap. Columns set to currency data type may wrap, but using 
\keyref{enter} will finish editing the cell (unless you're using the cell
editor dialog box). If you insert a newline character in the cell
edit dialog box (for any data type), the type for that column will 
be converted to \qt{string}.

To edit or view an entry in a column
with the \qt{string} data type, double-click on the relevant cell or 
select the cell and use \menu{edit.edit_cell} to open the cell editor 
dialog box (see \sectionref{sec:celleditor}). You can now scroll through
the cell contents.

\section{Cell Editor}\label{sec:celleditor}

To open the cell editor dialog box (see \figureref{fig:celleditor})
double-click on the required cell, which must be in a column with a 
string data type. Alternatively, select the cell (of any type) and use 
\menu{edit.edit_cell}.

\FloatFig[htbp]
 {fig:celleditor}
 {%
   \includeimg
    [
      scale=0.75,
      alt={Image of cell editor dialog}
    ]
    {images/celleditor.png}
 }
 {Cell Editor Dialog}

Remember that the contents of the cell should be \LaTeX\ code, so be
careful if you use any of \TeX's special characters. Also, see the
section on verbatim text (\sectionref{sec:verbatim}) if you haven't 
already read it.
Once you have made your edits, click on \widget{okay}
to update the database. To discard the edits, click 
\widget{cancel}.

If you've used \sty{datatool}, you will probably know that 
if you want a paragraph break in your cell entries you need to use 
\gls{DTLpar}, but with \app{datatooltk} you don't need to worry about it 
as blank lines in an entry will automatically be converted behind 
the scenes. Note that redundant blank lines will be removed. Leading
and trailing spaces are ignored when writing the \ext{dbtex} file,
but they will still be present in the cell editor until the file is
saved and reloaded (see \sectionref{sec:spaces}).

\begin{important}
If you use \sty{datatool}'s \gls{DTLwrite},
\gls{DTLsaverawdb} or \gls{DTLprotectedsaverawdb} commands to
overwrite your file, you will lose any pretty-printing spaces or
comments in your code.
\end{important}

\section{Header Dialog}\label{sec:editheader}

Each column has a title, a uniquely identifying label
and an associated type. The type can be one of:
\gls{header.type.string}, 
\gls{header.type.int}, 
\gls{header.type.real} or 
\gls{header.type.currency}.
The type is automatically detected from the column data,
but can be changed using the \menu{edit.column.edit_header}
menu item or by double-clicking on the column header which opens the 
header dialog box (see \figureref{fig:headerdialog}). The label corresponds 
to the label used to identify the column in commands such
as \gls{DTLforeach} and will be trimmed if the 
\switch{auto-trim-labels} setting is on. The title is used in 
commands like \gls{DTLdisplaydb}. See \sectionref{sec:preferences} for 
currency mappings. If the title field is left blank, it will be assigned 
the same value as the label.

\FloatFig[htbp]
 {fig:headerdialog}
 {%
   \includeimg
   [
     scale=0.75,
     alt={Image of header dialog box}
   ]
   {images/headerdialog.png}
 }
 {Header Dialog}

In \gls{gui} mode, column headers show the
title. If you move the mouse over the column header, you will see the
label and type displayed in a tooltip (see 
\figureref{fig:headerdetails}).

\FloatFig[htbp]
 {fig:headerdetails}
 {%
   \includeimg
    [
      scale=0.75,
      alt={Image of main window showing header tooltip}
    ]
    {images/headerdetails.png}
 }
 {Header Details Shown in Tooltip}

\chapter{Tools}\label{sec:tools}

There are currently two tools available: sort (see \sectionref{sec:sort})
and shuffle (see \sectionref{sec:shuffle}). These both reorder the rows of the
database and can be invoked either from the \menu{tools} menu
or from the command line (as long as
you have also loaded a database using \switch{in} or one of the
import options). If you use both \switch{sort} and \switch{shuffle}
in the command line invocation, sort will always be performed first, 
regardless of the option order.

\section{Sorting the Data}\label{sec:sort}

Although you can sort data in \sty{datatool} using \gls{DTLsort}, 
it's far more efficient to sort it in \app{datatooltk}. So instead of doing, say,
\begin{codebox}
\gls{input}\marg{mydata.dbtex}\comment{loads database \manmsg{labelled} `data' from file `mydata.dbtex'}
\gls{DTLsort}\marg{Surname}\marg{data}\comment{sort data on `Title' field}
\comment{Later in the document:}
\gls{DTLdisplaydb}\marg{data}\comment{display data in tabular environment}
\end{codebox}
It's better to run, say,
\begin{terminal}
\app{datatooltk} \switch{in} mydata.dbtex \switch{sort} Surname \switch{output} mydata-sorted.dbtex
\end{terminal}
(Remember that this defaults to letter sorting for strings. Use
\switch{sort-locale} to sort according to a locale.)
Then in the document, just load \filefmt{mydata-sorted.dbtex}:
\begin{codebox}
\gls{input}\marg{mydata-sorted.dbtex}
\comment{Later in the document:}
\gls{DTLdisplaydb}\marg{data}\comment{display data in tabular environment}
\end{codebox}
or, if you have the shell escape enabled you can used \TeX's 
\gls{write18} mechanism:
\begin{codebox}
\cmd{immediate}\cmd{write}18\marg{\app{datatooltk} \switch{in}
mydata.dbtex \switch{sort} Surname 
\switch{output} mydata-sorted.dbtex}
\codepar
\gls{input}\marg{mydata-sorted.dbtex}
\comment{Later in the document:}
\gls{DTLdisplaydb}\marg{data}\comment{display data in tabular environment}
\end{codebox}

\begin{important}
Beware of the security implications of enabling the shell escape.
\end{important}

If you have \app{arara} version 4.0+, there's a rule for \app{datatooltk}:
\begin{codebox}
\araraline{datatooltk: \marg{input: mydata.dbtex, sort: Surname,
\araracont{output: mydata-sorted.dbtex}}}
\araraline{pdflatex}
\end{codebox}
Incorporating \app{datatooltk} into the document build in this way
is safer than enabling the unrestricted shell.

\begin{information}
If the original data is in an \gls{sql} database, it's even more efficient to 
do the sorting in the SELECT statement when you import the data (see 
\sectionref{sec:importsql}).
\end{information}

A database can be sorted according to a particular column
in either ascending or descending order. In batch mode, this is done
with the \switch{sort} option, as shown above, where the sort column is identified
by the column's unique label. If the label is preceded by \code{-} then
descending order is used (for example, \code{\switch{sort}
-Surname}). If the label is preceded by \code{+} (or has
no prefix) then ascending order is used. When comparing strings there are 
two modes: letter (compare character codes) or locale-sensitive
(use the alphabet for the given locale).
For letter comparisons you 
can also use \switch{sort-case-sensitive} for case-sensitive comparisons
and \switch{sort-case-insensitive} for case-insensitive comparisons.
The default is case-insensitive. The locale comparisons are
typically case-insensitive. The treatment of accented characters depends
on the locale's rule.

In \gls{gui} mode, sorting is
done using the \menu{tools.sort} menu item which opens
the \dialog{sort} dialog box (see \figureref{fig:sortdialog}).

\FloatFig[htb]
 {fig:sortdialog}
 {%
   \includeimg
    [
      scale=0.75,
      alt={Image of sort dialog box}
    ]
    {images/sortdialog.png}
 }
 {Sort Dialog}

Select the column you wish to sort by from the drop-down list of column
titles, and check the appropriate radio button for ascending or descending
sort. If the column has the string data type, you also need to specify
what type of comparison you want to use. For a letter (character code)
comparison, select the \widget{sort.letter} box, which will enable
the \widget{sort.case_sensitive} box. For a locale comparison,
select the \widget{sort.locale} box, which will enable the
locale selector.

If the column type has a numerical type, the entries will be sorted via 
a numerical comparison (10 is greater than 2) and the string
options are ignored. If the column type is a string type,
any numerical entries will be sorted via a character comparison (\qt{10}
comes before \qt{2}).

\begin{example}{Sorting Data}{ex:sortingex}

Consider the data shown in \figureref{fig:sortex1} and reproduced in 
\tableref{tab:sortex1}.

\begin{table}[htbp]
\caption{Original Data}\label{tab:sortex1}
\centering
\begin{tabular}{ll}
Book & \code{\gls{pounds}5.99}\\
Video Game & \code{\gls{euro}20.00}\\
Pen & \code{\gls{pounds}3.00}
\end{tabular}
\end{table}

The first column has a string data type and the second has 
a currency data type. Sorting in ascending order on the second column,
will sort numerically on just the number. The currency symbol is ignored
(see \tableref{tab:sortex2}).
If the type of the second column is changed from currency to string, and the
sort is redone, the order is now based on a string comparison that
includes the currency symbol (see \tableref{tab:sortex3}).

\FloatFig[htbp]
 {fig:sortex1}
 {%
   \includeimg
    [
      scale=0.75,
      alt={Image of example database}
    ]
    {images/sortex1.png}
 }
 {Original Data}

\begin{table}[htbp]
\caption{Data Sorted on Second Column (Currency Comparison)}
\label{tab:sortex2}
\centering
\begin{tabular}{ll}
Pen & \code{\gls{pounds}3.00}\\
Book & \code{\gls{pounds}5.99}\\
Video Game & \code{\gls{euro}20.00}
\end{tabular}
\end{table}

\begin{table}[htbp]
\caption{Data Sorted on Second Column (String Comparison)}
\label{tab:sortex3}
\centering
\begin{tabular}{ll}
Video Game & \code{\gls{euro}20.00}\\
Pen & \code{\gls{pounds}3.00}\\
Book & \code{\gls{pounds}5.99}
\end{tabular}
\end{table}

Remember that \app{datatooltk} doesn't have any knowledge of currency conversions.
In this example it would be better to have a column of real numbers
containing the price in a single base currency. (In fact, it would
be better to store the original data in a spreadsheet or database
and just use \app{datatooltk} in batch mode.)
\end{example}

\section{Shuffling the Data}\label{sec:shuffle}

Shuffling involves randomly changing the order of the rows. This can be
performed either by the \switch{shuffle} command line option or
the \menu{tools.shuffle} menu item. You can change
the seed used by the random number generator with
\switch{seed} or through the 
\dialog{preferences} dialog box (see \sectionref{sec:preferences}).
The method used to shuffle data has changed since version 1.6.
If you need to use the old method (for example, if you
have set a seed with an existing document), then use
\code{\switch{compat} 1.6} when invoking \app{datatooltk}. For example

\begin{terminal}
datatooltk \switch{compat} 1.6 \switch{seed} 2000 \switch{shuffle} infile.dbtex \swch{output} outfile.dbtex
\end{terminal}
The newer version is more efficient.

\begin{example}{Shuffling Rows}{ex:shuffleex}

Consider the database shown in \figureref{fig:shuffleex1}. This database
has three columns. The first is a question, the second is the 
corresponding answer (optional) and the third is a number indicating 
the question level.  For example, 1 could correspond to easy and 2 could 
correspond to medium difficulty.

\FloatFig[htbp]
 {fig:shuffleex1}
 {%
  \includeimg
   [
      scale=0.75,
      alt={Image of example database}
   ]
   {images/shuffleex1.png}
 }
 {Shuffle Example}

Now suppose I want to write an assignment sheet that has one randomly
selected question of level 1 and two randomly selected questions of
level 2. Let's suppose the file name is \filefmt{data.dbtex}
and the database label is \qtt{problems}. Then I can run \app{datatooltk}
in batch mode using:
\begin{terminal}
\app{datatooltk} \switch{shuffle} \switch{in} data.dbtex \switch{output} data-shuffled.dbtex
\end{terminal}
Remember to use \switch{seed} if you don't want a different
ordering every time you run that command. For example:
\begin{terminal}
\app{datatooltk} \switch{seed} 2013 \switch{shuffle} \switch{in} data.dbtex \switch{output} data-shuffled.dbtex
\end{terminal}
This shuffled database can now be loaded in my document:
\verbatiminput{../tests/test-shuffle.tex}
What if I want all the easy questions listed first? This requires 
some modifications to the code as shown below:
\verbatiminput{../tests/test-shuffle-level.tex}

Now, the \gls{DTLforeach} loop just stores the row numbers of the
required questions in two lists, corresponding to the two different levels.
Then each list is iterated through and the corresponding row is fetched
using \gls{dtlgetrow}. Extending this example to accommodate an
arbitrary number of levels is left as an exercise for the reader.

Remember that if you have the shell escape enabled when you run \LaTeX\ you
can invoke \app{datatooltk} in your document \emph{before} you load the database:
\begin{codebox}
\cmd{immediate}\gls{write18}\marg{\app{datatooltk} \switch{in} data.dbtex \switch{seed} 2013 \switch{shuffle} 
\switch{output} data-shuffled.dbtex}
\codepar
\gls{input}\marg{data-shuffled.dbtex}
\end{codebox}
\end{example}

\section{Sorting and Shuffling}\label{sec:sortshuffle}

As mentioned earlier, if you specify both \switch{sort} and
\switch{shuffle}, the sorting will always be performed first, 
regardless of the option order, but why would you want to sort the data
if you're going to shuffle it? Consider the command invocation:
\begin{terminal}
\app{datatooltk} \switch{shuffle} \switch{in} \meta{in-file} \switch{output} \meta{out-file}
\end{terminal}
Every time you run this command, you will get a different ordering.
If, however, you set a seed for the random generator, for example:
\begin{terminal}
\app{datatooltk} \switch{seed} 2013 \switch{shuffle} \switch{in} \meta{in-file} \switch{output} \meta{out-file}
\end{terminal}
You will always get the same random ordering \emph{provided the original 
data in \meta{in-file} has remained unchanged}. If you want to 
modify the shuffled data in your document and save it to the original 
file \meta{in-file} using \gls{DTLsaverawdb}, the ordering in 
that file will change, so the next time you shuffle it, you'll get 
a different ordering, even if you use the same seed. If you sort
first on a unique label, that will ensure the shuffle has the same 
starting point (unless you add or remove rows).

\begin{example}{Sorting and Shuffling}{ex:sortshuffle}

Suppose you have a database of exam questions and you want to keep
track of the year in which each question was last used. (To make life
easier, let's identify the academic year \qt{2012/13} as 2013, the
academic year \qt{2013/14} as 2014, etc.) Let's further suppose the
database of questions is in a file called \filefmt{mth-101.dbtex}
and the database label is \qtt{problems} (see 
\figureref{fig:sortshuffleex}). The database 
contains a column with the label \qtt{Label}, which
uniquely identifies an exam question, a column with the label
\qtt{Question} that contains the exam question, a column
with the label \qtt{Answer} that contains the answer and
an integer column with the label \qtt{Year} that contains the
exam year in which that question was last used. (A zero entry means the
question hasn't been used.)

\FloatFig[htbp]
 {fig:sortshuffleex}
 {%
   \includeimg
   [
     scale=0.75,
     alt={Image of example database}
   ]
   {images/sortshuffleex1.png}
 }
 {Sort and Shuffle Example}

Now suppose the exam requires five questions to be randomly selected from 
this database, but must not include any question used in the past three 
years. So the exam \LaTeX\ document needs to load in a shuffled
version of \filefmt{mth-101.dbtex}, use the first five questions that don't
have a year set in the past three year range, set the year for the
selected questions to the current exam year, display the questions (and
optionally the answers for the solution sheet), and at the end of the
document, overwrite \filefmt{mth-101.dbtex} so that it now has 
a record of this year's exam questions.

There are two problems. Firstly, if the process is to be automated
with a call to \app{datatooltk} \switch{shuffle} followed by 
a \LaTeX\ call, a different set of problems will be selected
on each run, even with the same seed. To overcome this, a sort on 
the \qtt{Label} column needs to be done before the shuffle:
\begin{terminal}
\app{datatooltk} \switch{sort} Label \switch{seed} 2013 \switch{shuffle} \switch{in} mth-101.dbtex \continueline \switch{output} mth-101-shuffled.dbtex
\end{terminal}
\ContExplan
This way the shuffle always starts from the same ordering.

The second problem occurs if you edit the database such that you add or
remove rows. This will change the initial conditions, even with the sort.
If you add or remove rows, you need to accept that the document may well
end up with a different selection of questions, which is okay if
you haven't finalised the exam, but it means that some of the questions
will be identified as having been used in that exam year from 
a previous run but are now no longer selected. In order to make them
available for the next year, if they haven't been selected but have had the
year set to this year, the year needs to be cleared.

To solve this, once you have selected the maximum required number of
questions, don't break out of the loop, as was done earlier (see
\sectionref{sec:shuffle}). Instead, for the rest of the loop, if the exam
year is set to the current year, clear it.
\verbatiminput{../tests/test-exam.tex}

\begin{information}
Since this overwrites the \sty{datatool} file, you will
lose any pretty-printing spaces or comments you may have done in 
\app{datatooltk}'s cell editor dialog.
\end{information}

\end{example}

\section{Plugins}\label{sec:plugins}

   Plugins are usually associated with a particular template 
   (see \sectionref{sec:templates}) and provide a convenient way of
   adding a row of data to the currently selected database.
   Typically when a plugin is run it will add a new row of data 
   if no row is selected, otherwise it will allow you to edit the selected row.

   \begin{important}
   You must have Perl installed to use the plugins (see 
   \sectionref{sec:preferences}). 
   \end{important}

   \begin{warning}
   The plugin is sent the database information when you start each instance of 
   the plugin, so if you change the database in \app{datatooltk} while 
   a plugin is running there may be unexpected results. 
   Wait until the plugin has finished 
   (usually by clicking on \widget{okay} or 
   \widget{cancel}) before you make any further edits to the database.
   \end{warning}


\subsection{The people Plugin}\label{sec:peopleplugin}

   The \plugin{people} plugin is designed for use with databases created
   using the \template{people} template.

\begin{example}
 {Creating a new Database of People with a Plugin}
 {ex:peopleplugin}

   Suppose you create a new database using the \template{people}
   template. This creates a database with the following fields:
   ID, Title, Surname, Forename, Address, Telephone and Email, as
   illustrated in \figureref{fig:expeople}.

\FloatFig
 {fig:expeople}
 {%
   \includeimg
    [
      scale=0.75,
      alt={Image of main window}
    ]
    {images/peopletemplate.png}
 }
 {Database Created From people Template}

   Having created this database, I can just use the 
   \menu{edit.row} menu to insert rows and then edit each entry,
   but suppose I want to automatically increment the associated ID
   for each person. I can do this using the \plugin{people} plugin 
   that corresponds to this template via the \menu{tools.plugins} menu.

   If a row is currently selected, this plugin will allow you to edit
   the data for that row. Otherwise, it will allow you to insert a new
   row. For a new row of data, the \plugin{people} plugin will open
   the dialog box shown in \figureref{fig:expeopleplugin}.

\FloatFig
 {fig:expeopleplugin}
 {%
   \includeimg
    [
      scale=0.75,
      alt={Image of people dialog}
    ]
    {images/peopleplugin.png}
 }
 {The people Plugin Dialog}

   After entering the data, I can click on \widget{okay}
   and a new row of data is added to the database (see 
   \figureref{fig:peoplepluginex1}). Note that the plugin has converted newline
   characters in the address into \glssymbol{nlbksl}. The ID has automatically 
   been inserted.

\FloatFig
 {fig:peoplepluginex1}
 {%
   \includeimg
    [
      scale=0.75,
      alt={Image of main window}
    ]
    {images/peoplepluginex1.png}
 }
 {A New Row of Data}

   Since the \plugin{people} plugin only adds or modifies a single
   row at a time, if you no longer require an entry, you can delete the
   unwanted row using \menu{edit.row.remove_row}.
\end{example}

\subsection{The datagidx Plugin}\label{sec:datagidxplugin}

   The \sty{datagidx} package creates its own custom database 
   to store terms, symbols and acronyms. The \template{datagidx} template
   will create a database that contains \sty{datagidx}['s] required 
   fields. There are a lot of fields, some of which are reserved for
   \sty{datagidx}['s] private use. The \plugin{datagidx} plugin, 
   available via the \menu{tools.plugins} menu, provides 
   a convenient interface to add or edit entries. If no row is selected,
   the plugin will create a new row. If a row is selected,
   the plugin will allow you to edit or remove the row. Since the 
   \plugin{datagidx} plugin can modify other rows at the same time
   (for example, if you set a parent entry or cross-reference)
   it's recommended that you use the \plugin{datagidx} plugin to remove an
   entry (via the \widget{plugin.remove_entry} button) rather than 
   using \menu{edit.row.remove_row}.

\begin{example}
 {Creating a Database with the datagidx Template}
 {ex:datagidxplugin}

   The new database (created via \menu{file.new_from_template}) is
   shown in \figureref{fig:datagidxplugin1}. The default name of the database
   is \qt{\plugin{datagidx.default_name}}. You can change it 
   as required, but don't call it \qt{datagidx} as the \sty{datagidx} 
   package creates a database with that name for its private use.
   Once this database has been created, the \plugin{datagidx} plugin will 
   open the dialog box shown in \figureref{fig:datagidxplugin2}.

\FloatFig[htbp]
 {fig:datagidxplugin1}
 {%
   \includeimg
    [
      scale=0.75,
      alt={Image of main window}
    ]
    {images/datagidxplugin1.png}
 }
 {A Database Created From the datagidx Template}

\FloatFig[htbp]
 {fig:datagidxplugin2}
 {%
   \includeimg
    [
      scale=0.75,
      alt={Image of datagidx plugin dialog}
    ]
    {images/datagidxplugin2.png}
 }
 {The datagidx Plugin Dialog}

   Since many of the fields are often duplicated (for example, the 
   \optfmt{Name} field is often the same as the \optfmt{Text} field)
   if you first enter the name in the \plugin{datagidx.Name}
   field, when you move the focus to another field, default entries will 
   be added to most of the empty fields. For example, in 
   \figureref{fig:datagidxplugin3} I typed \qt{bird} in the
   \plugin{datagidx.Name} field and then moved the cursor to
   the \plugin{datagidx.Description} field. This automatically
   filled in default values for the \plugin{datagidx.Label},
   \plugin{datagidx.Sort},
   \plugin{datagidx.Text},
   \plugin{datagidx.Short},
   \plugin{datagidx.Long},
   \plugin{datagidx.Plural},
   \plugin{datagidx.ShortPlural} and
   \plugin{datagidx.LongPlural} fields.
   Since this is the first entry, there are no options for the
   \plugin{datagidx.Parent},
   \plugin{datagidx.See} and 
   \plugin{datagidx.SeeAlso} fields. (The last two are hidden in
   \figureref{fig:datagidxplugin3} as the \plugin{datagidx.crossref}
   button is unchecked.)

\FloatFig[htbp]
 {fig:datagidxplugin3}
 {%
   \includeimg
    [
      scale=0.75,
      alt={Image of datagidx plugin dialog}
    ]
    {images/datagidxplugin3.png}
 }
 {Most Fields Are Auto-Filled From the Name Field}

   When I click on \widget{okay}, a new row is
   added to the database (see \figureref{fig:datagidxplugin4}). Note that
   I didn't specify a parent for this entry so the parent has
   been given the value \gls{@dtlnovalue}, which ensures it will work
   correctly when the \sty{datagidx} package tests if the parent entry
   is null.

\FloatFig[htbp]
 {fig:datagidxplugin4}
 {%
   \includeimg
    [
      scale=0.75,
      alt={Image of main window}
    ]
    {images/datagidxplugin4.png}
 }
 {New Row Added to the Database}

   If I use the \plugin{datagidx} plugin to create a new row,
   there are now options available in the 
   \plugin{datagidx.Parent}, 
   \plugin{datagidx.See} and 
   \plugin{datagidx.SeeAlso} fields 
   (see \figureref{fig:datagidxplugin5}).

\FloatFig
 {fig:datagidxplugin5}
 {%
   \includeimg
    [
      scale=0.75,
      alt={Image of datagidx plugin dialog}
    ]
    {images/datagidxplugin5.png}
 }
 {Parent Field Lists Other Entry Labels}

   In \figureref{fig:datagidxplugin5} I have set the parent to 
   \optfmt{bird}. When the new row is added, the plugin automatically adjusts
   the \optfmt{bird} entry to include the new \optfmt{duck} label as one of its
   children (see \figureref{fig:datagidxplugin6}).

\FloatFig
 {fig:datagidxplugin6}
 {%
   \includeimg
    [
      scale=0.75,
      alt={Image of main window}
    ]
    {images/datagidxplugin6.png}
 }
 {Child Entry Automatically Adjusted For Parent Entry}


   It's also possible to cross-reference another entry. There are two ways
   of cross-referencing an entry: (1) using \plugin{datagidx.See}
   which redirects the reader to a synonym that has the location list;
   (2) using \plugin{datagidx.SeeAlso} which in addition to the
   location list refers the reader to one or more related topics. (See the
   \sty{datagidx} section of the \sty{datatool} user manual
   for further details.) To enable
   either form of cross-referencing, make sure the
   \plugin{datagidx.crossref} button is selected. This will 
   display extra options, shown in \figureref{fig:datagidxplugin7}.

\FloatFig
 {fig:datagidxplugin7}
 {%
   \includeimg
    [
      scale=0.6,
      alt={Image of datagidx plugin dialog}
    ]
    {images/datagidxplugin7.png}
 }
 {Cross-Referencing Entries}

   Either select the \plugin{datagidx.See} button and choose
   the synonym from the drop-down box next to it, or select the 
   \plugin{datagidx.SeeAlso} button and select the related 
   cross-reference from the drop-down box to the right and either click on
   \plugin{datagidx.add_seealso} to append it to the 
   \plugin{datagidx.SeeAlso} list or click on 
   \plugin{datagidx.remove_seealso} to remove it from the list.
   For example, in \figureref{fig:datagidxplugin7} I've added the \optfmt{chicken}
   and \optfmt{turkey} entries to the \plugin{datagidx.SeeAlso}
   list. (Assuming I've already added the \optfmt{chicken} and \optfmt{turkey}
   entries before defining this new entry.)


   Once I've enter all my terms, I can sort the data according to 
   the \plugin{datagidx.Sort} column. (Recall \sectionref{sec:sort}.)
   Now let's suppose I save this sorted database to a file
   called \filefmt{datagidx-test.dbtex}. I can now load it into 
   a \LaTeX\ document as follows:

\verbatiminput{../tests/datagidx-doc.tex}
Ensure you have at least version 2.15 of the \sty{datatool} bundle.
\end{example}

\subsection{Comparison of glossaries and datagidx}\label{sec:comparison}

   If you're interested in the comparative efficiency between
   using \sty{glossaries} and \sty{datagidx}, I performed
   a test with 100 entries randomly selected from a dictionary.
   The entries were listed in a file called \filefmt{entries} in
   the form:

\begin{codebox}
\gls{newterm}\marg{minnow}
\gls{newterm}\marg{running board}
\gls{newterm}\marg{diamant\'e}
\end{codebox}

   First, let's look at a document that uses \sty{datagidx} with \gls{newgidx}:

\verbatiminput{../tests/datagidx/test-datagidx.tex}

   In general you need three \LaTeX\ runs to compile a \sty{datagidx}
   document. In this case, you actually only need to do it twice
   since there are no location lists.


   Now let's test a \sty{datagidx} document where \app{datatooltk} does the
   sorting. First, we need to generate a \ext{dbtex}
   file that corresponds the same set of entries. This can be done
   with the following document:

\verbatiminput{../tests/datagidx/test-savedbtex.tex}

   This just converts the entries listed in \filefmt{entries.tex}
   into the appropriate database file, simulating having entered
   the terms using \app{datatooltk}['s] \plugin{datagidx} plugin. The file is saved
   as \filefmt{index.dbtex}. Remember that this data only needs
   to be sorted when you add a term. This can either be done
   in \app{datatooltk}['s] \gls{gui} mode or it can be done in batch mode:

\begin{terminal}
\app{datatooltk} \switch{in} index.dbtex \switch{sort} Sort \switch{output} index-sorted.dbtex
\end{terminal}

   This creates a file called \filefmt{index-sorted.dbtex}.
   This file can be loaded into a document as follows:

\verbatiminput{../tests/datagidx/test-loaddb.tex}

   Now let's look at the \sty{glossaries} package. Since the terms
   have been defined using \gls{newterm}, I've defined a command
   that will convert this into an equivalent \gls{newglossaryentry}.
   Some of the entries have accents in their name, which
   \sty{datagidx} automatically strips when generating the default 
   label, so I've added a quick way of generating an
   analogous accent-free label and sort key that can be used with \gls{newglossaryentry}. Here's the document:

\verbatiminput{../tests/datagidx/test-glossaries.tex}

   In order to compare them, I used \app{arara} with 
   the Linux \app{time} command. In each case, the \optfmt{clean}
   directive is used at the start to ensure the tests start without
   any auxiliary files. Since there are no location lists, only
   two \LaTeX\ calls are used on each example. If there were
   location lists, the \sty{datagidx} examples would both
   need a third \LaTeX\ call.

   \begin{information}
   The \sty{datagidx} package doesn't generate a location with \gls{glsadd}
   or \gls{glsaddall}, whereas \sty{glossaries} does.
   I've suppressed the location list in the \sty{glossaries}
   example to produce an equivalent document.
   \end{information}

   Remember that with the example that uses \filefmt{index-sorted.dbtex},
   \app{datatooltk} needs to sort the database whenever a new entry is 
   added to the database. Assuming that all possible required
   entries have been added to the database, we just need one
   sort operation:

\begin{terminal}
\app{datatooltk} \switch{in} index.dbtex \switch{sort} Sort \switch{output} index-sorted.dbtex
\end{terminal}
   Invoking this with the Linux \app{time} command gives:

\begin{verbatim}
real	0m0.296s
user	0m0.466s
sys	0m0.033s
\end{verbatim}

   Now \app{arara} can be run on each of the three test documents 
   (via the \app{time} command). The result from the first test
   that uses \sty{datagidx} and \gls{newgidx}. The results are:

\begin{verbatim}
real	0m13.801s
user	0m13.925s
sys	0m0.076s
\end{verbatim}

  \app{arara} records the total time taken as 13.39 seconds.


  The next test uses \sty{datagidx} and \gls{loadgidx}. 
  The result is:

\begin{verbatim}
real	0m2.643s
user	0m2.775s
sys	0m0.065s
\end{verbatim}

  \app{arara} records the total time taken as 2.23 seconds.


  The third test uses \sty{glossaries}. The result is:

\begin{verbatim}
real	0m1.156s
user	0m1.307s
sys	0m0.069s
\end{verbatim}

  \app{arara} records the total time taken as 0.75 seconds.

% TODO add example with bib2gls

   Using \sty{glossaries} is clearly faster than using
   \sty{datagidx}. In the case of \gls{loadgidx},
   \sty{glossaries} is approximately three times faster. In the 
   case of \gls{newgidx}, \sty{glossaries} is approximately 
  18 times faster. If a third \LaTeX\ run was required
  for the location lists with \gls{newgidx}, using \sty{glossaries} would be approximately 
  27 times faster (with only two \LaTeX\ runs and one \app{makeglossaries} run).


   If you're interested to know how this compares with \gls{makenoidxglossaries} instead of \gls{makeglossaries}, here's the revised 
   \sty{glossaries} code:

\verbatiminput{../tests/datagidx/test-glossaries-noidx.tex}

   The result is:

\begin{verbatim}
real	0m2.463s
user	0m2.596s
sys	0m0.065s
\end{verbatim}

  \app{arara} records the total time taken as 2.06 seconds.
  This is slightly quicker than the second \sty{datagidx} test.


\chapter{Importing Data}\label{sec:import}

   Data can be imported from \gls{csv} files (see 
   \sectionref{sec:importcsv}), \gls{sql} databases (see 
   \sectionref{sec:importsql}) or from files that can be imported with 
   the \sty{probsoln} package's \gls{loadallproblems} command
   (see \sectionref{sec:importprobsoln}). In the case of the first
   two, \appname\ can automatically convert \TeX's special characters if the
   \switch{map-tex-specials} command line option is used or the
   \widget{preferences.tex.map} option has been selected in the 
   \dialog{preferences} dialog box (see \sectionref{sec:preferences}).
   Both the column title and label will be obtained from the appropriate data 
   header.  The title will have any mappings applied (if set). The label will 
   have forbidden content (control sequences and the standard set of special 
   characters) removed. In the case of \gls{csv} files or spreadsheets
   imported without headers, default values will be used.


   Note that data can't be exported back to any of those formats.


\section{Import CSV Data}\label{sec:importcsv}

   Data can be imported from a \gls{csv} file using
   the \switch{csv} command line option or (in \gls{gui} mode)
   using the \menu{file.import.importcsv} menu item. The default
   separator is a \gls{comma} and the default delimiter is the \gls{dbqt}
   character. These can be changed using the \switch{csv-sep} and 
   \switch{csv-delim} command line options or in the 
   \dialog{preferences} dialog box (see \sectionref{sec:preferences}).

   If the \gls{csv} file has a header row, you must make sure 
   the \switch{csvheader} option is used or the 
   \widget{preferences.csv.hasheader} option is checked in the
   \dialog{preferences} dialog box.
   If the \gls{csv} file has no header row, you must make sure 
   the \switch{nocsvheader} option is used or the 
   \widget{preferences.csv.hasheader} option is unchecked in the
   \dialog{preferences} dialog box.


   Make sure that the \gls{csv} file encoding is correctly set
   before importing. This can be done from the \switch{csv-encoding}
   command line option or in the 
   \dialog{preferences} dialog box. The encoding
   of the \TeX\ (\ext{dbtex}) file is independent of the \gls{csv}
   encoding.

\begin{example}{Importing Data From a CSV File}{ex:importcsv}

   Consider the \gls{csv} file shown below:

\verbatiminput{../tests/test-para.csv}
   This has a cell with multiple lines. When it's imported into
   \app{datatooltk}, the paragraph break is converted to \gls{DTLpar}.
   However, this isn't visible when you look at the file in 
   \gls{gui} mode (see \figureref{fig:excsvpara}).

\FloatFig
 {fig:excsvpara}
 {%
   \includeimg
    [
      scale=0.75,
      alt={Image of main window showing cell with paragraph break}
    ]
    {images/csvpara.png}
 }
 {Paragraph Breaks Appear as a Single Blank Line}

   Note that the redundant second blank line in the \gls{csv} file
   has gone as multiple blank lines are replaced by a single
   \gls{DTLpar}.
\end{example}

\section{Import Spreadsheet Data}\label{sec:importspread}

   Data can be imported from an Excel \ext{xls} file (via the 
   Apache POI library \url{http://poi.apache.org/})
   or an Open Document Spreadsheet (via the jOpenDocument Library
   \url{http://www.jopendocument.org/}) using the
   \switch{xls} or \switch{ods} command line options, respectively.
   Alternative, in \gls{gui} mode you can use the 
   \menu{file.import.importspread} menu item.
   Note that the \ext{xlsx} is currently unsupported.

 When using the command line,
   you additionally need to specify the sheet index (starting from 0) or the 
   sheet name using the \switch{sheet} command line option. If you are 
   using the \gls{gui}, after you've selected the spreadsheet file 
   from the file selector dialog, you need to select the required sheet 
   name.
Importing data from a spreadsheet uses the same header row option and 
   \TeX\ mapping settings as the import \gls{csv} function. 
   So if the sheet doesn't have a header row, you need to use
   \switch{nocsvheader} or uncheck the
\widget{preferences.csv.hasheader} button 
   in the \dialog{preferences} dialog box. Note that 
   \ext{xlsx} files aren't supported.


   \begin{information}
   No formatting information is read when importing
   data from an Excel spreadsheet. It's up to you to explicitly
   add \LaTeX\ font commands when you include the data in your
   document.
   \end{information}


\section{Import SQL Data}\label{sec:importsql}

   Data can be imported from an \gls{sql} database using
   the \switch{sql} command line option or the 
   \menu{file.import.importsql} menu item. You additionally need to
   supply the database, port, prefix, host, user name and password. In batch 
   mode, you can use the command line options \switch{sqldb},
   \switch{sqlport}, \switch{sqlprefix}, \switch{sqlhost} and
   \switch{sqluser}. Although you can specify the password with 
   \switch{sqlpassword}, this isn't recommended as it isn't secure. 
   If you don't use \switch{sqlpassword}, 
   you will be prompted for the password, where the text you enter 
   won't be visible. Note that in batch mode, the default action
   is to use the console to request the password. If there's no
console available, you need to use the \switch{noconsole-action}
   to determine what action to perform. See \sectionref{sec:introduction} for more details
   about command line options.


   In \gls{gui} mode, when you use \menu{file.import.importsql}
   the dialog box shown in \figureref{fig:sqlimportdialog} will be displayed,
   where you can enter the settings. In addition to the above named settings,
   you must also specify the \gls{sql} SELECT statement that identifies 
   the required data to import. (This manual assumes that if you have
   data in an \gls{sql} database, then you have a basic
   knowledge of \gls{sql} syntax.)

\FloatFig
 {fig:sqlimportdialog}
 {%
   \includeimg
    [
      scale=0.75,
      alt={Image of the SQL dialog box}
    ]
    {images/sqlimportdialog.png}
 }
 {SQL Import Dialog Box}


   For example, in \figureref{fig:sqlimportdialog} I want to import all
   data from the table called \qt{customers} in the \gls{mysql}
   database called \qt{myshop}. (I've created a user called
   \qt{shopadmin} with SELECT privileges for this database.) Once I've 
   entered this information,
   I then click on \widget{okay} and the password
   dialog box will appear (see \figureref{fig:sqlpassworddialog}).

\FloatFig
 {fig:sqlpassworddialog}
 {%
   \includeimg
    [
      scale=0.75,
      alt={Image of the password dialog box}
    ]
    {images/sqlpassworddialog.png}
 }
 {Password Dialog Box}

   Alternatively, I can use batch mode to import and save the data 
   from the command prompt:

\begin{terminal}
\app{datatooltk} \switch{output} customers.dbtex \switch{sql} "SELECT * FROM customers" \continueline
\switch{sqldb} myshop \switch{sqluser} shopadmin
Password:
\end{terminal}

   \ContExplan The password should be entered at the \gls{password.prompt}
   prompt. Remember that it's more efficient to get the \gls{sql}
   database to do any sorting. For example (assuming the table has 
   a column called \optfmt{Surname}):
\begin{terminal}
\app{datatooltk} \switch{output} customers.dbtex \switch{sql} "SELECT * FROM customers ORDER BY \continueline
Surname" \switch{sqldb} myshop \switch{sqluser} shopadmin
Password:
\end{terminal}

\section{Import probsoln Data}\label{sec:importprobsoln}

   The \sty{probsoln} package allows you to define problems (and 
   optionally their solutions) using \gls{newproblem} or the
   \env{defproblem} environment. \app{datatooltk} can load a file
   containing these definitions and convert the \sty{probsoln} data
   into a \sty{datatool} database containing three or four columns
   with keys: \gls{probsoln.label}, 
   \gls{probsoln.question} and
   \gls{probsoln.answer}. If the imported data contains
   multiple \sty{probsoln} data sets, the fourth column has the
   key \gls{probsoln.set} and contains the dataset label. 
   You can import one of these
   files using the \switch{probsoln} command line option or (in
   \gls{gui} mode) using the \menu{file.import.importprobsoln}
   menu item. If you have a large number of problems, you can speed things
   up by setting the initial capacity to that number. (If the initial
   capacity is smaller than the total number of problems, the hash map
   used to store the data will have to be enlarged whenever the current
   capacity is exceeded.)

   \TeX\ is a difficult language to parse, so \app{datatooltk} uses the
   \gls{texparserlib} to help gather the data from the imported 
   file. Earlier versions of \app{datatooltk} used \LaTeX\ but this is no longer 
   required. This import function is governed by the \TeX\ file encoding
   and the \widget{preferences.tex.stripsolnenv} settings 
   (see \sectionref{sec:preferences}).


   \begin{information}
   If any problems require arguments the default values 
   will be used.
   \end{information}

\begin{example}
 {Importing Data From a probsoln File}
 {ex:importprobsoln}

   Consider the file called \filefmt{prob-mixed.tex} that contains the 
   following:

\verbatiminput{../tests/prob-mixed.tex}

   This contains a mixture of \gls{newproblem} and
   \env{defproblem}. It also has comments and spaces to make
   the code more readable. As can be seen in \figureref{fig:importps}
   these are now still in the import (whereas in older versions they were lost).

\FloatFig[hbt]
 {fig:importps}
 {%
   \includeimg
    [
      scale=0.75,
      alt={Image of the main window}
    ]
    {images/importps.png}
 }
 {Pretty Printing and Comments are No Longer Lost When Importing Data from probsoln}

\end{example}

   The problem defined with the unstarred version of \gls{newproblem}
   has a different result depending on whether or not the 
   \widget{preferences.tex.stripsolnenv} setting is on. The 
   normal definition of this command (as provided by \sty{probsoln})
   wraps the solution (given in the final argument) in the 
   \env{solution} environment. This is stripped when the setting is
   on, otherwise it's included in the \qtt{Answer} column.

   \seealsorefs{sec:shuffle,sec:sortshuffle}

\chapter{Templates}\label{sec:templates}

   Templates that come with \app{datatooltk} are located in the \filefmt{resources/templates}
   subdirectory of the \app{datatooltk} installation directory. You can also write
   your own templates and store them in the user templates directory 
   (see \sectionref{sec:writingtemplates}). Each template defines a set
   of column headers. To create a new database with a particular
   set of column headers, use the \menu{file.new_from_template}
   menu item, which opens the dialog box shown in \figureref{fig:templatedialog}.

\FloatFig
 {fig:templatedialog}
 {%
   \includeimg
    [
      scale=0.75,
      alt={Image of template dialog}
    ]
    {images/templatedialog.png}
 }
 {New From Template Dialog}

   The \app{datatooltk} application comes with the following templates: 
   \template{datagidx} (creates a database with the same structure as
   used by the \sty{datagidx} package) and \template{people} (creates
   a database suitable for storing records about people, including
   columns for forenames, a surname, title and address.) For example,
   \figureref{fig:peopletemplate} shows a database created from the 
   \template{people} template.

\FloatFig
 {fig:peopletemplate}
 {%
   \includeimg
    [
      scale=0.75,
      alt={Image of main window}
    ]
    {images/peopletemplate.png}
 }
 {New Database Created from people Template}


   Rows can now be added to this database using the \menu{edit.row}
   menu or via corresponding plugins (see \sectionref{sec:plugins}).


\section{Writing a Template File}\label{sec:writingtemplates}

   If you want to write your own template, you need to create an XML file
   and store it in a subdirectory  of the \app{datatooltk} 
   user properties directory (see \sectionref{sec:preferences}) called 
   \filefmt{templates}. You will need to create this directory, if it doesn't
   already exist. For example, on a UNIX-like system, the user
   template directory will be \homefilefmt{.datatooltk/templates/}.
   The template file must have the extension \ext{xml} for it to be
   listed in the \dialog{template} dialog box. (The base
   name of the file is used in the list.)

   The template file must have one \xmltagfmt{datatooltktemplate} element. This
   element may contain one or more \xmltagfmt{header} elements. Each 
   \xmltagfmt{header} element must contain one \xmltagfmt{label} element and 
   optionally one \xmltagfmt{title} and/or one \xmltagfmt{type} element.


   The \xmltagfmt{label} element contains the uniquely identifying header label.
   The \xmltagfmt{title} element contains the header title. If omitted, the
   title is set to the label, unless there is an entry in the resource
   dictionary file that matches \code{plugin.}\meta{template name}\code{.}\meta{label}, 
   in which case that property is used. The \xmltagfmt{type} element must be one of:
   \code{-1} (unknown type), \code{0} (string type),\code{1} (integer 
   type), \code{2} (real type) or \code{3} (currency type). If omitted the 
   type is set to \code{-1}.

\begin{example} 
 {Creating a Products Template}
 {ex:productstemplate}

   Suppose I want to write a template to create a database
   for a list of products. The database needs three columns: one for
   the product name, one for the product code and one for the product price.
   The name should be a string, the price column could either be set
   to \qt{real} if you don't need to worry about the currency unit
   or \qt{currency} if you need a currency unit for each product.
   Let's suppose that the code must be an integer.
   Here's a template file (the price column is set to \qt{real}
   rather than \qt{currency}):

\begin{compactcodebox}
\xmltagfmt{datatooltktemplate}
  \xmltagfmt{header}
    \xmltagfmt{label}Name\xmltagfmt{/label}
    \xmltagfmt{type}0\xmltagfmt{/type}
  \xmltagfmt{/header}
  \xmltagfmt{header}
    \xmltagfmt{label}Code\xmltagfmt{/label}
    \xmltagfmt{type}1\xmltagfmt{/type}
  \xmltagfmt{/header}
  \xmltagfmt{header}
    \xmltagfmt{label}Price\xmltagfmt{/label}
    \xmltagfmt{type}2\xmltagfmt{/type}
  \xmltagfmt{/header}
\xmltagfmt{/datatooltktemplate}
\end{compactcodebox}
\end{example}

\chapter{Application Properties}\label{sec:preferences}

When \app{datatooltk} is run, either in batch or \gls{gui} mode, 
the application settings are read in from the user properties file, 
if it exists. Any command line options override those settings.
If \app{datatooltk} is run in \gls{gui} mode, the application
properties are saved on exit. They are not saved in batch mode, so
if you want to change the default settings for batch mode, without
having to use the applicable command line option, you will need to run
\app{datatooltk} in \gls{gui} mode to set them as required.

The user properties directory depends on the operating system.
On Windows, it is a folder called \file{datatooltk-settings}
in the folder given by the Java system property \code{user.home}.
This is usually the user's home folder but in some versions of
Java this can be \code{\%userprofile\%}.
On other operating systems, the user properties directory is
called \file{.datatooltk} and is in the user's home directory.
Alternatively, set the environment variable \gls{DATATOOLTK}
to the directory of your choice.

In \gls{gui} mode, the settings can be changed using
\menu{edit.preferences}. This opens the \dialog{preferences} dialog
box, which has the following tabs.

\sectionentry{preferences.general}

\FloatFig[htbp]
 {fig:prefgeneral}
 {%
   \includeimg
    [
      scale=0.75,
      alt={Image of the general tab of the preferences dialog box}
    ]
    {images/pref-general.png}
 }
 {General Tab}

In the \widget{preferences.general} tab (\figureref{fig:prefgeneral}) you can specify the start up directory. (The default 
directory when you first load, save or import data via the 
\menu{file} menu.) You can set this to your home directory,
the \gls{cwd}, the directory you last used on the previous run
of \app{datatooltk} or you can specify a directory of your choice.

In this tab you can also specify the seed for the random number generator 
(equivalent to \switch{seed}) and whether or not to automatically
strip leading and trailing spaces from database and column labels
(equivalent to \switch{auto-trim-labels} and 
\switch{noauto-trim-labels}).


     The initial capacity can be increased to speed up loading or importing.
     Ideally it's best to keep it around the typical size of your databases.
     If it's too big you can run out of memory. If it's too small, the storage
     has to be enlarged every time the current capacity is exceeded. The
     minimum allowed value is 10.

\sectionentry{preferences.csv}

\FloatFig[htbp]
 {fig:prefcsv}
 {%
   \includeimg
    [
      scale=0.75,
      alt={Image of the csv tab of the preferences dialog box}
    ]
    {images/pref-csv.png}
 }
 {CSV Tab}

In the \widget{preferences.csv} tab (\figureref{fig:prefcsv}) you can specify the \gls{csv} settings. Some of
these settings are also used by the spreadsheet import functions.

If the separator is a tab character, select the 
\widget{preferences.csv.tabsep} radio button. Otherwise
select the (\widget{preferences.csv.sep})
\widget{preferences.csv.sepchar} radio button
and enter the character in the neighbouring text box. Set the delimiter
in the \widget{preferences.csv.delim} field, and
check the \widget{preferences.csv.strictquotes} button
if you want to ignore any data that hasn't been delimited.

% TODO correct this
     You can also specify the escape character. This character can be used to 
     escape the delimiter character if it occurs in any of the fields. Since the
     escape character is a \gls{bksl} by default, this means
     that if the data contains any (La)TeX commands the backslash will need
     to be doubled. This conflict can be avoided by changing the 
     \gls{csv} escape character to something else (that doesn't occur 
     in your data). To change it, select the (\widget{preferences.csv.esc}) 
     \widget{preferences.csv.escchar} button and enter the
     character in the neighbouring text box. Alternatively, you can suppress 
     the \gls{csv} escape character, in which case the delimiter character can't occur within 
     the data. To do this, select the \widget{preferences.csv.noesc}
     button.  The character encoding can be changed through the 
     \widget{preferences.csv.encoding} drop-down box.

Check the \widget{preferences.csv.hasheader} button if your 
     \gls{csv} files have a header row otherwise uncheck it, and
     check the \widget{preferences.csv.skipemptyrows} button
     to skip empty rows. To skip a set number of rows, change the
     \widget{preferences.csv.skiplines} value to the required number 
     of rows. (Use 0 to switch off this function.) Note that the skip lines
     function is independent of the skip empty rows. If you have set the 
     skip lines value to, say, 3 then the first 3 lines are automatically 
     skipped regardless of whether or not they have any content. The check for 
     empty rows won't start until the next row (row 4, in this case).
     The header, skip empty rows and skip lines settings are also used by
     the spreadsheet import functions.

\sectionentry{preferences.sql}

\FloatFig[htbp]
 {fig:prefsql}
 {%
   \includeimg
    [
      scale=0.75,
      alt={Image of the SQL tab of the preferences dialog box}
    ]
    {images/pref-sql.png}
 }
 {SQL Tab}

In the \widget{preferences.sql} tab (\figureref{fig:prefsql}), you can specify the \gls{sql} connection information.
Enter the host name and port number the \gls{sql} server is 
running on in the \widget{preferences.sql.host} and 
\widget{preferences.sql.port} fields. Currently, the only
available prefix is \qtt{jdbc:mysql://}, which is the JDBC driver for
\gls{mysql}. If you are using another driver or \gls{sql} 
database, you'll have to add the relevant library to the \filefmt{lib}
directory and add it to the class path used by \file{datatooltk.jar}.
Enter the name of the database you want to connect to in the 
\widget{preferences.sql.database} field and the associated
user name in the \widget{preferences.sql.user} field.
If you want the password wiped from memory as soon as a connection
has been made, make sure the \widget{preferences.sql.wipe}
box has been selected.

\sectionentry{preferences.tex}

\FloatFig[htbp]
 {fig:preftex}
 {%
   \includeimg
    [
      scale=0.75,
      alt={Image of the tex tab of the preferences dialog box}
    ]
    {images/pref-tex.png}
 }
 {TeX Tab}


The \widget{preferences.tex} tab (\figureref{fig:preftex}) governs
\TeX\ related settings.
\LaTeX\ is no longer used to help \app{datatooltk} import data from 
a \sty{probsoln} dataset. Instead the \gls{texparserlib} is used
to parse \ext{tex} files.

The default file encoding for \TeX\ files (including \ext+{dbtex}
and \ext+{dtltex} files opened and saved by \app{datatooltk}) is set
in this tab in the \widget{preferences.tex.encoding} dropdown list.
Note that the \ext{dbtex} and \ext{dtltex} file formats allow a
special comment on the first line that identifies not only the file
format but also the encoding. For example, a \gls{DBTEX3} file that's
UTF-8 encoded should start with the line:
\begin{codebox}
\% DBTEX 3.0 UTF-8
\end{codebox}
The encoding identifier may also be an \sty{inputenc} option name,
such as \code{utf8}.

As from \sty{datatool} v3.0, data can be input into a document using
\gls{DTLread}, which includes support for \ext+{dbtex} and
\ext+{dtltex} file formats: \gls{DBTEX3}, \gls{DBTEX2}, \gls{DTLTEX3}
and \gls{DTLTEX2}. Earlier versions of \sty{datatool} only
support \gls{DBTEX2} and \gls{DTLTEX2}. These formats are described in
the \sty{datatool} v3.0+ manual.

The \gls{DBTEX3} format is the only one that supports the datum
format, where the textual representation, numeric value, currency
symbol and data type are stored separately, which means that the textual
representation doesn't need to by parsed by \sty{datatool} in
numeric contexts. This is therefore the default for new databases.

The \ext{dtltex} formats can have their elements converted to datum
format while the file is read by \gls{DTLread}, depending
\sty{datatool}['s] \opt{store-datum} setting. However, this setting
doesn't affect the \ext{dbtex} formats.

\begin{important}
No format allows verbatim or category code changes to occur in the
data. The entries should only contain document commands, not
internal \code{@}-commands or \LaTeX3 syntax. (The entries can, of
course, contain commands which use internals in their definition,
but any arguments must assume the default category codes.)
\end{important}

In general it's best to update \sty{datatool} at the same time as
updating \app{datatooltk}, but if you have an older version of
\sty{datatool} that doesn't support the newer file formats, then you
can change the default format in the \widget{preferences.tex.format}
selector to \gls{DBTEX2} to allow for backward compatibility.  This
will be used for new databases or databases obtained by importing
from other sources.  If you save a database that was loaded from a
\ext{dbtex} or \ext{dtltex} file, then the original format will be
the default unless the \widget{preferences.tex.override_format}
checkbox is selected.

If a database is saved in the \gls{DBTEX3} format, then you can
specify whether or not to save the datum elements. In general, it's
best to use the datum format for decimals and currency. It can also
be useful for integers if their numeric values need to be used in a
calculation in the document. However, if the integer values are
unique identifies that need to be queried, it's better not to use
the datum format as it can interfere with queries. There's usually
no need to save strings in the datum format unless a column or row
contains mixed types that need parsing. (For example, with column or row
aggregate actions.) As with a column containing unique integer value
identifiers, a column containing unique string identifiers that need
to be queried are best saved without the datum format.

To indicate whether or not to used the datum format when saving to a
\gls{DBTEX3} file for entries, select the applicable radio button:
\widget{preferences.tex.save_datum.none} (don't use the datum format
for any entries), \widget{preferences.tex.save_datum.all} (use the
datum format for all non-null entries), 
\widget{preferences.tex.save_datum.header_type} (determine whether
or not to use the datum format according to the column's data type),
or \widget{preferences.tex.save_datum.entry_type} (determine whether
or not to use the datum format according to the entry's data type,
which may be different to the column type).

For the last two options, the applicable type (column or entry)
determines whether or not to use the datum format. If either of
those options are set, the following checkboxes will become visible:

\begin{itemize}
\item \widget{preferences.tex.save_datum.type.int} use the datum
format for integers, if this box is selected;
\item \widget{preferences.tex.save_datum.type.decimal} use the datum
format for decimals, if this box is selected;
\item \widget{preferences.tex.save_datum.type.currency} use the datum
format for currencies, if this box is selected;
\item \widget{preferences.tex.save_datum.type.string} use the datum
format for currencies, if this box is selected.
\end{itemize}

\begin{information}
Null entries are omitted when saving the data.
\end{information}

When importing \sty{probsoln} data, you may find it more convenient
to strip any instances of the \env{solution} environment, 
particularly the implicit use of this environment by the
unstarred version of \gls{newproblem}. You can now choose whether or 
not to omit \code{\cbeg{solution}} and \code{\cend{solution}}
by selecting the \widget{preferences.tex.stripsolnenv} button.
If checked, any instances of \env{solution} contained within 
definitions (provided by \gls{newproblem} or \env{defproblem}) 
will be removed.

In the \widget{preferences.tex} tab you can also specify whether or
not to map \TeX\ special characters when you import data from
non-\TeX\ sources.  If you want the mapping, make sure the
\widget{preferences.tex.map} box is checked. The mapping table and
buttons are hidden if \widget{preferences.tex.map} is unchecked. If
it is checked, the performed mappings are listed in the table in the
tab. To add another mapping, click on the
\widget{preferences.tex.add_map} button, which opens the dialog box
shown in \figureref{fig:addmap}. 

\FloatFig[htbp]
 {fig:addmap}
 {%
   \includeimg
    [
      scale=0.75,
      alt={Image of the add mapping dialog box}
    ]
    {images/addmap.png}
 }
 {Add Mapping Dialog}

    To remove a mapping, select the unwanted mapping and click on 
    \widget{preferences.tex.remove_map}. To edit a mapping,
    select the mapping and click on \widget{preferences.tex.edit_map}.

\sectionentry{preferences.currencies}

\FloatFig[htbp]
 {fig:prefcurrencies}
 {%
   \includeimg
    [
      scale=0.75,
      alt={Image of the currencies tab of the preferences dialog box}
    ]
    {images/pref-currencies.png}
 }
 {Currencies Tab}


If you want to identify a column as a currency type, you 
must make sure that \app{datatooltk} recognises the \LaTeX\ command to typeset
your currency. Known currency commands are listed in the 
\widget{preferences.currencies} tab (\figureref{fig:prefcurrencies}). If you add any currencies
to the list, remember to add them in your document as well with 
\gls{DTLnewcurrencysymbol}.

\sectionentry{preferences.display}

\FloatFig[htbp]
 {fig:prefdisplay}
 {%
   \includeimg
    [
      scale=0.75,
      alt={Image of the display tab of the preferences dialog box}
    ]
    {images/pref-display.png}
 }
 {Display Tab}

The \widget{preferences.display} (\figureref{fig:prefdisplay}) tab governs user interface settings.
The default font used in cell entries is a monospaced font. This can be
changed using the \widget{preferences.display.font} drop-down menu.
You can also set the font size in the 
\widget{preferences.display.fontsize} field. By default, each
string cell has a maximum of four lines visible in the main window.
(Real and integer columns only have a single line visible.) This
number can be changed in the \widget{preferences.display.cellheight}
field. Each column has a default width that depends on the data type
for that column. The values are listed in the 
\widget{preferences.display.cellwidths} area. These can be changed
as required.

The \gls{LF} refers to the way the graphical interface
is rendered. You can use the drop-down menu to
select a different \gls{LF}, but you need to restart \app{datatooltk} 
for the change to take effect. For example, \figureref{fig:prefdisplay}
shows the \qt{Metal} \gls{LF} whereas
\figureref{fig:prefdisplaynimbus}
shows the same window but with the \qt{Nimbus} \gls{LF}.

\FloatFig[htbp]
 {fig:prefdisplaynimbus}
 {%
   \includeimg
    [
      scale=0.75,
      alt={Image of the display tab of the preferences dialog box
with Nimbus Look and Feel}
    ]
    {images/pref-display-nimbus.png}
 }
 {Display Tab (Nimbus Look and Feel)}

\sectionentry{preferences.language}

\FloatFig[htbp]
 {fig:preflanguage}
 {%
   \includeimg
    [
      scale=0.75,
      alt={Image of the language tab of the preferences dialog box}
    ]
    {images/pref-language.png}
 }
 {Language Tab}


The \widget{preferences.language} tab (\figureref{fig:preflanguage}) controls the localisation settings.
The language used by the manual accessed via \menu{help.manual}
can be set from the \widget{preferences.language.helpset} drop-down 
list. The language used in the messages, menu items, buttons and 
\gls{gui} labels can be set from the 
\widget{preferences.language.dictionary} drop-down 
list. Note that you have to restart \app{datatooltk} for these changes to take 
effect.

% TODO add information about the numeric settings

\sectionentry{preferences.plugins}

\FloatFig[htbp]
 {fig:prefplugins}
 {%
   \includeimg
    [
      scale=0.75,
      alt={Image of the plugins tab of the preferences dialog box}
    ]
    {images/pref-plugins.png}
 }
 {Plugins Tab}

In order to use \app{datatooltk} plugins, you must have Perl installed (and the
Perl Tk module). If the Perl executable is on your path, you can just
specify it as \qtt{perl} in the \widget{preferences.plugins.perl}
field of the \widget{preferences.plugins} tab (\figureref{fig:prefplugins}). If it's not on your
path, you will have to specify the full path name in this tab. You can
use the ellipsis button to browse your filing system.

\chapter{Help Menu (GUI Mode)}
\label{sec:helpwindows}

\menudef{menu.help.about}

The \menu{help.about} menu item shows the \InlineMsgDef[\appname]{about.title}
dialog with version details. In batch mode this information can be
obtained with the \switch{version} switch.

% Copied from https://github.com/nlct/texjavahelp/blob/main/src/doc/helpinterface.tex
\menudef{menu.help.manual}

The application's manual is available as either a \gls{pdf}
document, which can be viewed outside of the application, or as a
set of \gls{html} files which can be viewed within the application
via the \menu{help.manual} menu item. This will open the primary
help window (\sectionref{sec:primaryhelp}), but some dialog boxes
may also have a \inlineglsdef{action.help} button that will open a secondary help
dialog (\sectionref{sec:secondaryhelp}).

Both the primary help window and the secondary help dialog windows
have a panel that shows a page of the manual (a
\inlineglsdef{index.help-page}).  Note that \qt{page} in this
context refers to the \gls{html} file displayed in the help window, which
typically contains a section, and doesn't relate to the page numbers
in the \gls{pdf}. The \gls{html} index page is obtained from the
same source code as the \gls{pdf} index page, but the locations are
converted from a \gls{pdf} page number to the \gls{html} page title
(preceded by the marker \gls{symbol.location_prefix}).

Although the \dgls{help-page} is not editable, for some versions of
Java, the caret is visible when the page has the focus, and
the caret can be moved around using the arrow keys on your keyboard.

\menudef{index.menu.helppage}

The \gls{index.menu.helppage} (see \figureref{fig:helppagepopup})
can be activated on the current \dgls{help-page} for both the
primary and secondary help windows. The mouse press to show a popup
menu is typically the right mouse button, but this may not be the
case for all operating systems.  The popup menu can also be
activated using the context menu \keys{\keyref{contextmenu}} key if the
\dgls{help-page} has the focus.  The menu has the following items.

\FloatSubFigs
{fig:helppagepopup}
 {
   {fig:helpframepopup}
   {\includeimg
     [alt=
      {Primary help window popup menu}
     ]{images/helppagepopup}%
   }
   {},
   {fig:helpdialogpopup}
   {\includeimg
     [alt=
      {Secondary help multi-page dialog popup menu}
     ]{images/helpdialogpopup}%
   }
   {},
   {fig:helpdialogsinglepopup}
   {\includeimg
     [alt=
      {Secondary help single-page dialog popup menu}
     ]{images/helpdialogsinglepopup}%
   }
   {}
}
[Help Page Popup Menus]
{Help Page Popup Menus: 
 \subfigref{fig:helpframepopup} Primary Help Window;
 \subfigref{fig:helpdialogpopup} Secondary Help Dialog
 with Multiple Topic Pages;
 \subfigref{fig:helpdialogsinglepopup} Secondary Help Dialog
 with Single Topic Pages}

\menudef{menu.helppage.view_image}

If the popup menu is activated over an image, the \menu{helppage.view_image}
item will open the \dialog{imageviewer} window (see
\sectionref{sec:helpimageviewer}) which can be used to enlarge the
image. This item will be disabled if the popup menu wasn't activated
over an image.

Where the popup menu was activated using the context menu
\keys{\keyref{contextmenu}} key, 
the position of the caret will determine
whether or not to enable this menu item.

\menudef{menu.helppage.home}

If the popup menu is activated on the primary help window
(\figureref{fig:helpframepopup}), this will
behave as the \menu{helpframe.navigation.home} menu item (which
switches the current page to the first page of the document).
This menu item is not available on secondary help windows.

\menudef{menu.helppage.reset}

If the popup menu is activated on the secondary help dialog
(\figuresref{fig:helpdialogpopup,fig:helpdialogsinglepopup}), this
will behave as the \menu{helpdialog.navigation.reset} menu item
(which switches the current page back to the relevant page or the
first in the applicable section of the dialog topic).
It will be disabled if the current page is the reset target page.
This menu item is not available on the primary help window.

\menudef{menu.helppage.up}

If the popup menu is activated on the primary help window or on the
secondary help window that has multiple pages
(\figuresref{fig:helpframepopup,fig:helpdialogpopup}), then this will behave
as the primary \menu{helpframe.navigation.up} or
secondary \menu{helpdialog.navigation.up} menu items. (That is, it
will move up a hierarchical level, if available.)
This menu item will be disabled if there is no parent page (or, for
secondary windows, no parent page within the topic set).

\menudef{menu.helppage.previous}

If the popup menu is activated on the primary help window or on the
secondary help window that has multiple pages
(\figuresref{fig:helpframepopup,fig:helpdialogpopup}), then this
will behave as the primary \menu{helpframe.navigation.previous} or
secondary \menu{helpdialog.navigation.previous} menu items.  (That
is, it will move to the previous page, if available.) This menu item
will be disabled if there is no previous page (or, for secondary
windows, no previous page within the topic set).

\menudef{menu.helppage.next}

If the popup menu is activated on the primary help window or on the
secondary help window that has multiple pages
(\figuresref{fig:helpframepopup,fig:helpdialogpopup}), then this will behave
as the primary \menu{helpframe.navigation.next} or
secondary \menu{helpdialog.navigation.next} menu items.
(That is, it will move to the next page, if available.)
This menu item will be disabled if there is no next page (or, for secondary
windows, no next page within the topic set).

\menudef{menu.helppage.historyback}

This will behave as the primary
\menu{helpframe.navigation.historyback} or secondary
\menu{helpdialog.navigation.historyback} menu items.
(That is, it will move back a page in the history list, if available.)
This menu item is in all the help page popup menus but will be
disabled if there is no page to go back to.

For the secondary help windows, it's possible to follow a link in
the current page to a page outside the topic set. The menu item can
take you back to the previously visited page viewed in that
secondary dialog window.

\menudef{menu.helppage.historyforward}

This will behave as the primary
\menu{helpframe.navigation.historyforward} or secondary
\menu{helpdialog.navigation.historyforward} menu items.
(That is, it will move forward a page in the history list, if available.)
This menu item is in all the help page popup menus but will be
disabled if there is no page to go forward to.


\section{The Primary Help Window}
\label{sec:primaryhelp}

The primary help window is the main help frame accessed via
\menu{help.manual}, which has a panel that shows a page of the
manual (a \dgls{help-page}). Links in the page and the \gls{gui}
navigation elements provide a way to switch to a different page.

There is a menu bar with items for navigation actions or adjusting
\gls{gui} settings. Some menu items are replicated as buttons in the
toolbar, which is split into different regions: navigation, lookup,
settings, and history. The forward, up and next navigation actions
can also be implemented by buttons in the lower navigation panel at
the bottom of the window.

\menudef*{menu.helpframe.navigation}

The \menu{helpframe.navigation} menu provides a way to move around
the document.
\Figureref{fig:navbuttons} shows the corresponding four navigation
buttons in the toolbar: \btn{helpframe.navigation.home} (go to the
start of the manual), \btn{helpframe.navigation.previous} (go to the
previous section), \btn{helpframe.navigation.up} (go to parent
section), and \btn{helpframe.navigation.next} (go to the next
section).

\FloatFig
{fig:navbuttons}
{\includeimg
 [alt=
   {
     [\entrytooltip{menu.helpframe.navigation.home} Button]
     [\entrytooltip{menu.helpframe.navigation.previous} Button]
     [\entrytooltip{menu.helpframe.navigation.up} Button]
     [\entrytooltip{menu.helpframe.navigation.next} Button]
   }
 ]{images/navbuttons}%
}
[Primary Help Window Navigation Buttons]
{Primary Help Window Navigation Buttons (Home, Previous, Up, Next)}

\menudef{menu.helpframe.navigation.home}

The \menu{helpframe.navigation.home} item, which is also available
as a button on the toolbar, will replace the current view with the
first page of the document.

\menudef{menu.helpframe.navigation.up}

The \menu{helpframe.navigation.up} item, which is also available
as a button on the toolbar, will replace the current view with the
parent page of the current hierarchical level. The item and button
will be disabled if there is no parent page (that is, if the current
page is the document's home page). The parent page may
also be the previous page if the current page is the first in its
current hierarchical level.

\menudef{menu.helpframe.navigation.previous}

The \menu{helpframe.navigation.previous} item, which is also available as
a button on the toolbar, will replace the current view with the
previous page. The item and button will be disabled if there is no
previous page. (That is, if the current page is the first
page of the document.)

\menudef{menu.helpframe.navigation.next}

The \menu{helpframe.navigation.next} item, which is also available as
a button on the toolbar, will replace the current view with the
next page. The item and button will be disabled if there is no
next page. (That is, if the current page is the last
page of the document.)

\Figureref{fig:search+index} shows the search and index buttons,
which may be used to lookup relevant pages.

\FloatFig
{fig:search+index}
{\includeimg
 [alt=
   {
     [\entrytooltip{menu.helpframe.navigation.search} Button]
     [\entrytooltip{menu.helpframe.navigation.index} Button]
   }
 ]{images/search+index}%
}
{Search and Index Buttons}

\menudef{menu.helpframe.navigation.search}

The \menu{helpframe.navigation.search} item, which is also available
as a button on the toolbar, will open the
\dialog{help.navigation.search} window (see
\figureref{fig:searchframe}), from which you can search the document
for a keyword. If any matches are found, the title of the relevant
page is shown as a hyperlink, which links to the start of the page.
The title is followed by a block of text where the search term was
found (which will be highlighted). Clicking on the block of text
should scroll to a nearby location in the relevant page.

\FloatFig
{fig:searchframe}
{\includeimg
 [alt=
   {image of search window showing search term highlighted in a paragraph}
 ]{images/searchframe}%
}
{Search Window}

\menudef{menu.helpframe.navigation.index}

The \menu{helpframe.navigation.index} item, which is also available
as a button on the toolbar, will open the index page in a separate
window (see \figureref{fig:indexframe}). You can also open the same
page in the help window at the end of the document. The separate index window
provides a way of navigating the document without having to keep
returning to the index page. Additionally, the index window has a
split page with links on the left to scroll the page to a letter
group.

If an indexed item is shown as a hyperlink, then that link will go
to the principle definition of that item. The indexed item may also
be followed by a list of pertinent locations that are preceded by
the symbol \gls{symbol.location_prefix}.

\FloatFig
{fig:indexframe}
{\includeimg
 [alt=
   {image of index window showing part of the document index}
 ]{images/indexframe}%
}
{Index Window}

\Figureref{fig:historybuttons} shows the history buttons.
Note that the forward button is greyed (disabled) because the
currently viewed page is at the end of the history list, so it's not
possible to go forward.

\FloatFig
 {fig:historybuttons}
 {%
   \includeimg
    [alt=
     {[\entrytooltip{menu.helpframe.navigation.history} Button]
      [\entrytooltip{menu.helpframe.navigation.historyback} Button]
      [\entrytooltip{menu.helpframe.navigation.historyforward} Button]
     }
    ]
    {images/historybuttons-annote}%
 }
 {History Buttons}

\menudef{menu.helpframe.navigation.history}

The \menu{helpframe.navigation.history} menu item, which is also
available as a button on the toolbar, opens the
\dialog{help.navigation.history} window,
(see \figureref{fig:historywindow}).

The current page has the title shown in bold and is preceded by
the symbol \gls{symbol.help.navigation.history.pointer}.
Select the required page and click on the
\gls{help.navigation.history.go} button.

\FloatFig
 {fig:historywindow}
 {\includeimg
   [alt={image of the help page history window}]
   {images/historyframe}%
 }
 {The Page History Window}

\menudef{menu.helpframe.navigation.historyback}

The \menu{helpframe.navigation.historyback} menu item, which is
also available as a button on the toolbar, will replace the current
view with the previously viewed page from this history list. The
item and button will be disabled if there is no previously viewed
page. 

\menudef{menu.helpframe.navigation.historyforward}

The \menu{helpframe.navigation.historyforward} menu item, which is
also available as a button on the toolbar, will replace the current view with the
next page in the history list. The item and button will be disabled if the
currently viewed page is at the end of the history list. 

\menudef*{menu.helpframe.settings}

The \menu{helpframe.settings} menu can be used to change the
graphical interface settings. These settings affect the primary and
secondary help windows, as well as some other related windows.
Note that this is separate from the main application settings.

\menudef{menu.helpframe.settings.decrease}

The \menu{helpframe.settings.decrease} item decreases the font
size by 1.

\menudef{menu.helpframe.settings.increase}

The \menu{helpframe.settings.increase} item increases the font
size by 1.

\menudef{menu.helpframe.settings.font}

The \menu{helpframe.settings.font} item opens the
\dialog{help_font_settings} dialog (see
\sectionref{sec:helpfontdialog}).

\menudef{menu.helpframe.settings.nav}

The \menu{helpframe.settings.nav} item opens the
\inlineglsdef{help_settings_nav.title} dialog. This governs the
lower navigation bar (see \figureref{fig:helplowernavbar}) along the
bottom of the primary help window, which has smaller previous, up and next
buttons.  These buttons by default have the corresponding page
titles next to them, but they will be truncated if they exceed the
limit. This limit can be changed with the
\widget{help_settings_nav.label_limit} widget. Alternatively, you
can hide the text by deselecting the
\widget{help_settings_nav.show_label} checkbox.

\FloatFig
{fig:helplowernavbar}
{\includeimg
 [alt=
   {Help page lower navigation bar}
 ]{images/helplowernavbar}%
}
{Help Page Lower Navigation Bar}

\section{Secondary Help Window}
\label{sec:secondaryhelp}

The secondary help windows are more minimalist and will only show
the relevant \dgls{help-page} or set of pages that are applicable to
the context that was used to open the secondary help window. If only
one page is applicable, there won't be a navigation tree, otherwise
the navigation tree will only show the applicable pages.

The search, history and index windows are unavailable, but it is
possible to move back and forward in the history list for the
current secondary help window. The topic page will be added to the
primary help window history but otherwise the page history lists
aren't shared between the help windows.

The secondary help windows are designed for use with modal dialogs
(that is, a window that blocks the main application window)
to provide help for the particular dialog. The primary help window
can't be accessed while a modal dialog is open so it will
automatically be closed when a secondary help window is opened.
You can re-open the primary help window once you have closed the
modal dialog.

\menudef*{menu.helpdialog.navigation}

The \menu{helpdialog.navigation} menu provides a way to move around
the topic pages.

\menudef{menu.helpdialog.navigation.reset}

The \menu{helpdialog.navigation.reset} item switches the current
page to the first page of the context topic. This menu item
will be disabled if the current page is the reset target page.

\menudef{menu.helpdialog.navigation.historyback}

The \menu{helpdialog.navigation.historyback} goes back to the
previously visited page. Note that the history is specific to the
current secondary help dialog instance and does not include the history
from the primary help window. This menu item will be disabled if
there is no page in the history list to go back to.

\menudef{menu.helpdialog.navigation.historyforward}

The \menu{helpdialog.navigation.historyforward} moves forward in the
history list, if applicable. This menu item will be disabled if
there is no page in the history list to go forward to.

The \btn{menu.helpdialog.navigation.previous}, 
\btn{menu.helpdialog.navigation.up} and
\btn{menu.helpdialog.navigation.next} buttons
(\figureref{fig:secondaryNavbuttons}) are only available if the
topic context contains multiple pages.

\FloatFig
{fig:secondaryNavbuttons}
{\includeimg
 [alt=
   {
     [\entrytooltip{menu.helpdialog.navigation.reset} Button]
     [\entrytooltip{menu.helpdialog.navigation.previous} Button]
     [\entrytooltip{menu.helpdialog.navigation.up} Button]
     [\entrytooltip{menu.helpdialog.navigation.next} Button]
   }
 ]{images/secondaryNavButtons}%
}
[Secondary Help Window Navigation Buttons for Multi-Page Topics]
{Secondary Help Window Navigation Buttons for Multi-Page Topics
  (Reset, Previous, Up, Next)}

\menudef{menu.helpdialog.navigation.previous}

The \menu{helpdialog.navigation.previous} menu item is only
available if there are multiple pages for the topic context and will
switch the current page with the previous page in the topic set.
This menu item will be disabled if the previous page is not within the
topic set.

\menudef{menu.helpdialog.navigation.up}

The \menu{helpdialog.navigation.up} menu item is only
available if there are multiple pages for the topic context and will
switch the current page with the parent page if it's within the topic set.
This menu item will be disabled if there is no parent page or if the
parent page is not in the topic set.

\menudef{menu.helpdialog.navigation.next}

The \menu{helpdialog.navigation.next} menu item is only
available if there are multiple pages for the topic context and will
switch the current page with the next page in the topic set.
This menu item will be disabled if the next page is not within the
topic set.

\section{Help Font Dialog}
\label{sec:helpfontdialog}

The \menu{helpframe.settings.font} item opens the
\inlineglsdef{help_font_settings.title} dialog (see
\figureref{fig:helpfontdialog}). Use the
\widget{help_font_settings.family} selector for the main body font
family and the \widget{help_font_settings.size} selector for the
main body font size. Icon characters, such as
\gls{symbol.help.navigation.history.pointer}, may not be available
for your preferred font family, so you can specify an alternative
with the \widget{help_font_settings.icon_font_family} selector. This
will only list fonts that support some commonly used icon
characters.

\FloatFig
{fig:helpfontdialog}
{\includeimg
 [alt=
   {Help page font dialog}
 ]{images/helpfontdialog}%
}
{Help Page Font Dialog}

Use the \widget{help_font_settings.keystroke_font_family} selector to
choose the font to show keystrokes (such as
\keys{\keyref{shift}}) and the
\widget{help_font_settings.mono_font_family} selector to choose the
font to display code fragments (such as \verb|% \ { } #|).

The document hyperlink style can also be changed with the 
\widget{help_font_settings.hyperlinks} 
\btn{help_font_settings.choose_colour} and
\btn{help_font_settings.underline} widgets.

The styles are applied to the primary help window, all secondary
help windows and related windows, such as the \dialog{help.navigation.history} 
or index windows.

\section{Image Viewer}
\label{sec:helpimageviewer}

The \menu{helppage.view_image} item in the \gls{index.menu.helppage}
for both the primary and secondary help windows will be enabled if
the \gls{index.menu.helppage} is activated over an image. The
\menu{helppage.view_image} item will open the image in the
\inlineglsdef{imageviewer.title} window.  If the image had alt text
specified, this will be displayed in the area above the image.

Within the \dialog{imageviewer} window, the image can be enlarged
using the \widget{imageviewer.magnify} spinner. The up and down
spinner controls go in steps of 25 (as opposed to the
\btn{menu.imageviewer.increase} and \btn{imageviewer.decrease}
action, which have an increment of 5). Alternatively, press the
shift key \keys{\keyref{shift}} and drag the mouse to select an area to
zoom in on. Be sure to keep the shift key down when you release the
mouse. If you change your mind, release shift before releasing the
mouse button. If the shift key isn't pressed when you initiate the
drag, dragging will scroll the image instead. Double-clicking the
mouse on the image will go back to the previous magnification.

\menudef{index.menu.imageviewer}

The \gls{index.menu.imageviewer} is a popup menu that can
be activated anywhere over the image in the \dialog{imageviewer}
window. The following menu items are available.

\menudef{menu.imageviewer.fit_to_width}

The \menu{imageviewer.fit_to_width} item will scale the image so
that it fits the window width. This action has a corresponding
button on the toolbar.

\menudef{menu.imageviewer.fit_to_height}

The \menu{imageviewer.fit_to_height} item will scale the image so
that it fits the window height. This action has a corresponding
button on the toolbar.

\menudef{menu.imageviewer.fit_to_page}

The \menu{imageviewer.fit_to_page} item will scale the image so
that it fits within the window area. This action has a corresponding
button on the toolbar.

\menudef{menu.imageviewer.increase}

The \menu{imageviewer.increase} item will increase the current
magnification. This action has a corresponding
button on the toolbar.

\menudef{menu.imageviewer.decrease}

The \menu{imageviewer.decrease} item will decrease the current
magnification. This action has a corresponding
button on the toolbar.

\menudef{menu.imageviewer.zoom_1}

The \menu{imageviewer.zoom_1} item will set the magnification factor to
100\%. This action has a corresponding
button on the toolbar.

\menudef{menu.imageviewer.zoom_2}

The \menu{imageviewer.zoom_2} item will set the magnification factor to
200\%.

\menudef{menu.imageviewer.zoom_5}

The \menu{imageviewer.zoom_5} item will set the magnification factor to
500\%.




\chapter{\MFUsentencecase{\glsentrytext{index.licence}}}
\label{sec:licence}

   \glsadd{index.licence}%
   \appname\ is licensed under the terms of the 
   \href{https://www.gnu.org/licenses/gpl-3.0.html}{GNU General
   Public License version 3 (GPLv3)}.
   \app{datatooltk} depends on the following third party libraries whose jar files are
   in the \filefmt{lib} directory:
   \begin{itemize}
   \item \TeX\ Parser Library \filefmt{texparserlib.jar}
   (GPL, \url{https://github.com/nlct/texparser});
   \item TeX Java Help \filefmt{texjavahelplib.jar}
   (GPL, \url{https://github.com/nlct/texjavahelp});
   \item JOpenDocument \filefmt{jOpenDocument-1.5.jar}
     (GPL, \url{https://jopendocument.org/});
   \item MySQL connector \filefmt{mysql-connector-java.jar}
     (GPLv2, \url{http://dev.mysql.com/downloads/connector/j/});
   \item Apache POI \filefmt{poi-5.2.5.jar}, \filefmt{poi-ooxml-5.2.5.jar}
   and \filefmt{poi-ooxml-full-5.2.5.jar}
    (\href{https://www.apache.org/licenses/LICENSE-2.0.html}{Apache 2}, \url{https://poi.apache.org/}).
   \end{itemize}


\listentrydescendents
 [title={Summary of \apptext{datatooltk} Switches}]
 {app.datatooltk}

\printmain
\printindex

\end{document}
